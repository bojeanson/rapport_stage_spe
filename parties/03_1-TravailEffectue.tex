Des travaux initiaux avaient été réalisés en Python par Samuel Charron. Il avait réalisé un plugin récupérant les signaux, construisant un classifieur naïf bayésien multinomial avec et permettant de classifier de nouveaux signaux. cependant les performances n'étaient pas suffisantes. N'étant pas formé au Python, j'ai préféré commencer mes travaux en utilisant le Java avec l'accord de Samuel. Je savais, en m'orientant vers le Java, qu'une fois que l'application obtiendrait de bonnes performances, j'aurais à implémenter son fonctionnement général en Python sous forme de plugin pour pouvoir l'intégrer à l'architecture de C-Radar.

\section{Démarche de travail}
    Ce projet s'inscrit parfaitement dans le type de projet R\&D. De ce fait, l'avancement est très difficile à planifier dans le temps. Surtout lorsque l'on ne connaît pas les différentes notions sous-jacentes au projet et qu'il y a une bonne part d'auto-formation avant de pouvoir développer une application.

    \subsection{Mes acquis à l'INSA}
        Les connaissances générales que j'avais en \textit{Data Science}, avant le début du stage, concernaient le \textit{Data Mining} en contexte \textbf{numérique} et étaient les suivantes :
        \begin{itemize}
            \item Concepts en analyse et normalisation de données : Analyse en Composantes Principales (ACP), centrage et réduction de données numériques ;
            \item Concepts d'apprentissage non-supervisé : méthodes de regroupement des données (Clustering : Classification Hiérarchique Ascendante, algorithmes des K-Means et EM) ;
            \item Base de l'optimisation : méthodes du gradient et de Newton, introduction à l'optimisation sous contraintes convexe ;
            \item Concepts d'apprentissage supervisé : méthodes pour la discrimination de données (Décision Bayésienne, Régression logistique, SVM linéaire) et notions de validation croisée.\\
        \end{itemize}

        Ce projet ne permet pas de mettre mes connaissances en apprentissage non-supervisé en avant. Cependant, mes notions d'apprentissage supervisé telles que : la démarche à suivre pour construire un classifieur, les concepts liés à la validation des performances (validation croisée) et la notion de sur-apprentissage ; ont été fort utiles.\\

        Mes connaissances en \textit{Text Mining} n'étaient pas suffisamment étoffées pour pouvoir dire tel classifieur est plus performant qu'un autre dans tel contexte (binaire ou multi-classe). En effet, ma formation (à l'INSA) est axée manipulation et traitement de données \textbf{numériques}. De ce fait, une formation en \textit{Text Mining} m'était donc indispensable avant de pouvoir commencer le développement d'une application.

    \subsection{Déroulement du stage}
        Ainsi, durant les deux premiers mois, j'ai exploré le domaine du \textit{Text Mining} et du \textit{Natural Language Processing} au travers de la bibliothèque de Stanford implémentée en Java (\textit{Stanford Natural Language Processing}). Conjointement, j'ai étudié les cours associés, et construit une première application Spring répondant aux contraintes évoquées en partie \ref{sec:ma_mission_chez_data_publica} (sauf le critère du langage). Le travail en ressortant est décrit en partie \ref{sec:travaux_realises_en_java}.\\

        Ensuite, lors du dernier mois, j’ai implémenté le comportement général de cette application sous la forme d’un plugin Python (partie \ref{sec:travaux_python}). Certains composants n'existaient pas en Python, je les ai donc ré-implémentés.