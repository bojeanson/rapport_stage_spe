\section{Bilan sur mes travaux}
    J'ai réussi à mettre au point un plugin Python fonctionnel, capable de classifier les signaux captés par C-Radar. Celui-ci obtient de meilleures performances que le précédent, ce qui était l'objectif, et celles-ci sont satisfaisantes. De plus, pour souligner l'aboutissement de ce stage, ce plugin a pu être déployé en production.\\

    Les principales difficultés rencontrés en première partie de stage sont liées à mon manque de connaissances en \textit{Text Mining} et \textit{Natural Language Processing}. Cependant, j'ai su passé outre grâce à un bon accompagnement de la part de mes tuteurs et grâce à de l’expérience acquise, notamment, lors du projet de type R\&D pour Orange Vallée au semestre 8.\\

    J'ai rencontré d'autres difficultés. Celles-ci liées au le manque d'outils de \textit{Naturel Language Processing} disponible pour la langue française. En effet, pour l'anglais les outils sont très bien aboutis et nombreux, mais pour le français, il est difficile d'en trouver. Je pense, notamment, au Lemmatiser qui n'existe pas pour le français en Python 3. D'autres outils existe mais en Python 2. (De grandes différences existent entre Python 2 et Python 3. Parfois, des wrapers permettant de faire le pond entre les deux, mais leur utilisations ne sont pas de bonnes pratiques.)\\

    Les pistes d'amélioration possibles sont les suivantes :
    \begin{itemize}
        \item Introduire un outils permettant de détecter la langue du signal (comme \href{https://tika.apache.org/}{Tika}) afin d'adapter les traitements en aval ;
        \item Utiliser un détecteur d'entités nommées \textit{Named Entity Recognizer} afin de gérer autrement les noms d'entreprises, de personnes qui sont généralement caractéristiques des signaux \textit{MONEY} (exemple : \og Data Publica a levée 10 m€ en janvier auprès du fond d'investissement français constitué du groupe Lagardère...\fg). Pour le moment ces noms propres sont rejeté comme feature car tous sont isolés n'étant pas les mêmes.
        \item Éventuellement, redéfinir la classe \textit{MONEY} en deux classes, comme suit, pour voir si le classifieur s'améliore :
        \begin{itemize}
            \item \textit{MONEY} : Déclaration de résultats financiers, de CA (chiffre d'affaire), de levée de fonds ;
            \item \textit{BUSINESS} : Déclaration de partenariat, de rachat ou d'aquisition d'entreprises, de \og contrat gagné\fg, etc.
        \end{itemize}
        \item Se tourner vers des outils statistiques comme le traducteur de Google pour exécuter la lemmatisation.\\
    \end{itemize}