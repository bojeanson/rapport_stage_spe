Ce stage a été une expérience très enrichissante tant sur le plan humain et relationnel, que sur le plan technique. Je souhaitais absolument découvrir l'univers de la start-up et c'est chose fait grâce à Data Publica et ses équipes.\\

Sur le plan humain, j'ai eu la chance de rencontrer des personnes extrêmement sympathiques et compétentes, mais aussi ouvertes à la discussion et toujours disponible. L'équipe de Data Publica m'aura toujours bien encadré. Que ce soit Samuel Charron (et Christian Frisch) pendant la première partie stage ou Clément Chastagnol (et Guillaume Lebourgeois) en deuxième partie, j'aurais constamment eu quelqu'un à mon écoute. En effet, lorsque je rencontrais une difficulté, j'avais, à chaque moment, quelqu'un de disponible à \og déranger \fg pour mettre à l'épreuve son expérience ou ses connaissances.\\

Un excellent stage sur le plan relationnel aussi. En effet, tout le monde m'a accordé du temps à mon arrivée pour se présenter, et me dire leurs tâches et responsabilités chez Data Publica. De plus, partager des déjeuners et des pots conviviales a permis de renforcer et aussi de créer de premiers liens avec les collègues.\\

J'ai également beaucoup aimé cet esprit de partage et ces discussions ouvertes, lors des réunions des équipes pour différents points.\\

Sur le plan technique, j'ai énormément progresser dans le domaine du \textit{machine learning} et énormément appris dans le domaine du \textit{text mining} que je ne connaissais pas. Tous ces concepts de normalisation de textes, le langage Python et tous ces modules disponibles (scikit-learn, numpy, nltk, etc). J'ai découvert une facette de la \textit{data science} que j'ignorais et qui m'a beaucoup plu.//

Ce stage aura donc été très riche, techniquement parlant, par la découverte de domaines inconnus et passionnants, mais aussi humainement parlant, par la rencontre de personne formidable et poussant constamment à la réussite et au dépassement de soi.

