Ce stage obligatoire s'effectue en fin de quatrième année. Au cours de ce stage l'étudiant devra mettre en pratique les connaissances acquises au cours de sa formation et devra approfondir son savoir-faire au sein de l'entreprise. Il faudra qu'à  la fin de son stage l'étudiant réalise un rapport écrit. La validation du stage dépend de la qualité du travail réalisé, du rapport et de la fiche d'évaluation du tuteur industriel.\\
Ce stage a eu lieu chez Data Publica, un des précurseurs de l'open data en France. Data Publica est une jeune start-up (fondée en juillet 2011) spécialisée dans les données entreprises, l'open data, le big data et la dataviz. Data Publica emploie quatorze personnes très dynamiques et compétentes. Data Publica développe C-Radar, un produit de vente prédictive construit sur une base de référence des entreprises françaises regroupant informations administratives, financières, web, réseaux sociaux et media. C-Radar est un concentré de technologies du big data (crawling, scraping ou encore machine learning).\\
Ma mission, chez Data Publica, est de construire une chaîne de traitement automatique, un plugin Python, récupérant une liste de documents textuels en entrée et fournissant, en sortie, une liste de ces documents labellisés selon leur catégorie d'intérêt (offre d'emploi, participation à des événements, nomination de personnel, levée de fond, etc).\\
Les tâches qui en découlent sont directement liées au \textit{Text Mining} et au \textit{Natural Language Processing} (aussi à l'\textit{Information Retrieval}) dans la découverte d'informations dans le contenu des documents. Des compétences en \textit{Machine Learning} sont également requises.\\
Ce stage s'est conclue de belle manière, puisque le plugin Python est fonctionnel et capable de classifier de tels documents.