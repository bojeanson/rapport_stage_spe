\documentclass[12pt,twoside]{scrreprt}
\usepackage[T1]{fontenc}
\usepackage[utf8]{inputenc}
\usepackage{lmodern}
\usepackage{textcomp}
\usepackage[francais]{babel, varioref}
\usepackage{graphicx}
\usepackage{listings}
\usepackage{xspace}
\usepackage{amsmath}
\usepackage{amssymb}
\usepackage{calc}
\usepackage{listingsutf8}
\usepackage{color}
\usepackage{xcolor}
\usepackage{afterpage}
\usepackage[style=verbose-note,backend=bibtex]{biblatex}
\usepackage{url}
\usepackage[top=2.1cm,bottom=2.2cm,left=2cm,right=2cm]{geometry}
\usepackage[final]{pdfpages}
\usepackage{subcaption}
\usepackage{amsmath}
\usepackage{array}
\captionsetup{compatibility=false}

\lstdefinelanguage{xml}{morestring=[b]",
                        morestring=[s]{>}{<},
                        morecomment=[s]{<?}{?>},
                        stringstyle=\color{black},
                        identifierstyle=\color{black},
                        keywordstyle=\color{cyan},
                        morekeywords={ROOT,S,NP,VP,ADVP,PRP\$,NN,NNP,NNS,JJ,DT,VBD,PP,PDT,TO,IN}% list your attributes here
                        }


\lstset{language=xml,
        escapeinside={K}{W},
        %basicstyle=\ttfamily,
        columns=fullflexible,
        commentstyle=\color{gray}\upshape
        tabsize=3,
        label=code:sample,
        frame=shadowbox,
        rulesepcolor=\color{gray},
        xleftmargin=20pt,
        framexleftmargin=15pt,
        numbers=left,
        numberstyle=\tiny,
        numbersep=5pt,
        breaklines=true,
        showstringspaces=false,
        basicstyle=\footnotesize,
        emph={food,name,price},emphstyle={\color{magenta}}
        }

\newcommand{\hlc}[2][yellow]{ {\sethlcolor{#1} \hl{#2}} }
\newcommand{\highlight}[2][yellow]{\mathchoice%
  {\colorbox{#1}{$\displaystyle#2$}}%
  {\colorbox{#1}{$\textstyle#2$}}%
  {\colorbox{#1}{$\scriptstyle#2$}}%
  {\colorbox{#1}{$\scriptscriptstyle#2$}}}

\definecolor{gray}{rgb}{0.4,0.4,0.4}
\definecolor{darkblue}{rgb}{0.0,0.0,0.6}
\definecolor{cyan}{rgb}{0.0,0.6,0.6}

% Pour sommaire cliquable
\usepackage{hyperref} % Créer des liens et des signets
\hypersetup{
colorlinks=true, %colorise les liens
breaklinks=true, %permet le retour à la ligne dans les liens trop longs
urlcolor= blue, %couleur des hyperliens
linkcolor= black, %couleur des liens internes
citecolor=black,  %couleur des références
}

% Fichier de bibliographie
\bibliography{parties/biblio}

\usepackage{templateINSA}
\initINSA

% Tirte centre
\renewcommand\infoBig{Baptiste O'Jeanson}
\renewcommand\infoSmall{Rapport de stage de spécialité}

% Titre bas
\title{Classification de signaux entreprises avec une approche machine learning}
\renewcommand\soustitre{Information retrieval, Text mining and Natural Language Processing}

% Auteurs
\author{
	\textbf{Étudiant:} Baptiste \bsc{O'Jeanson} \\
	\textbf{Maître de stage:} Christian \bsc{Frisch}\\
	\textbf{Entreprise:} Data Publica\\
	\textbf{Année:} 2014-2015\\
	\textbf{Dates:} du 01/06/2015 au 28/08/2015\\
	\textbf{Lieu:} Paris, France\\
	}

\begin{document}

% titleINSA : Page de garde
% #1 : descendre le titre du milieu (en mm)
% #2 : lien de l'image de fond
% #3 : décalage sur X de l'image de fond (en mm)
% #4 : décalage sur Y de l'image de fond (en mm)
% #5 : largeur de l'image de fond de #5 (en mm)
% #6 : Crédit de l'image de fond
 \titleINSA{0}{images/logo-data-publica.jpg}{0}{50}{220}{Image : \href{http://www.data-publica.com/}{\color{white}{http://www.data-publica.com/}}}


% Remerciements
\section*{ \begin{center} \textbf{ {\LARGE Remerciements} } \end{center} }
\phantomsection

Premièrement, j'aimerais remercier François Bancilhon, directeur général de Data Publica, et Christian Frisch, directeur technique de Data Publica, de m'avoir offert l'opportunité de découvrir le monde du travail en start-up. Je voudrais aussi les remercier de m'avoir fait confiance et de m'avoir confié une mission très intéressante.

\paragraph{}
Je voudrais ensuite remercier Samuel Charron, Clément Chastagnol et Guillaume Lebourgeois qui m'ont suivi durant mon stage et qui ont enrichi mes connaissances en informatique.

\paragraph{}
Je souhaiterais également remercier l'ensemble de l'équipe de développeurs et l'ensemble de l'équipe de marketing de m'avoir accueilli si chaleureusement.

\paragraph{}
Enfin, merci à mon tuteur de stage, Nicolas Malandain, pour sa disponibilité.
\addcontentsline{toc}{chapter}{Remerciements}

% Sommaire
\tableofcontents

% Parties
\chapter{Présentation de l'entreprise}
L’évolution des technologies et leurs usages ont fait exploser la quantité de données générées. Selon IBM, 2.5 milliard de gigabytes (GB) de données a été générée tout les jours de l'année 2012. De plus, cette quantité de données, double tout les deux ans. Cependant, seules 0,05\% de ces données sont analysées.\\

L’exploration ou la fouille de données (\og data mining \fg) consiste à en extraire des informations utiles, et ceci peut s’avérer très fructueux. La question principale qui se pose est de savoir comment utiliser intelligemment cette immense masse de données pour en tirer une plus-value ?\\
C’est le rôle des entreprises spécialisées dans l'exploitation de ces données.

\section{L'entreprise}
    La société Data Publica a été fondée en juillet 2011 par François Bancilhon et Christian Frisch, respectivement l'actuel directeur général et l'actuel directeur technique.

    \subsection{L'activité de Data Publica}
        Data Publica est un des précurseurs de l'open data en France. \textcolor{red}{Cette société , qui a bénéficiée d’investissements technologiques faits en 2010 dans le cadre d’un projet de R\&D, a été financée initialement par un groupe de business angels et le fonds d’amorçage \textbf{IT Translation}.}
        Data Publica est un start-up spécialisée dans les données entreprises, l'open data, le big data et la dataviz. C'est une société relativement jeune, axée R\&D. Son leitmotiv, alimenté par une équipe très dynamique et compétente, est la recherche constante du dépassement technique.

        \paragraph{}
            Historiquement, Data Publica ne faisait que de l'open-data. C'est-à-dire que la société se servait de données accessibles à tous (provenant d'institutions gouvernementales notamment) pour créer des jeux de données sur mesure pour des entreprises. Ainsi, la société s'est spécialisé dans l'identification des sources de données, leur extraction et leur transformation en données structurées.

        \paragraph{}
            Depuis quelques années, Data Publica se spécialise dans les données sur les entreprises française en dépit de son activité open-data qu'elle a progressivement mis de côté. Les services qu'elle propose ne sont plus tout-à-fait les mêmes. En effet, Data Publica réutilise les données open-data concernant les entreprises française dans son produit phare. Ce produit est lui même conçu pour les entreprises du B2B. Le produit est décrit en partie \ref{c_radar}.

        \paragraph{}
            Data Publica participe également à de nombreux projets de recherche français et européens tels que XDATA, Diachron ou Poqemon, en partenariat avec l'INRIA.

    \subsection{L'équipe de Data Publica}
        Data Publica emploie 14 personnes réparties en 2 équipes : une équipe commerciale (4 personnes) et une équipe technique (10 développeurs). Les deux équipes travaillent chacune dans son open-space. Pendant mon stage, j'ai été immergé au sein de l'équipe technique.

        \paragraph{L'équipe technique :}
            Elle est composé de 10 développeurs (ordonnés par ancienneté visible en figure \ref{fig:teamd_data_publica}) :
            \begin{itemize}
                \item Christian Frisch, directeur de l'équipe
                \item Thomas Dudouet, \textcolor{red}{Java / Back end développeur}
                \item Guillaume Lebourgeois, chef de produit C-Radar
                \item Samuel Charron, \textcolor{red}{Data scientist Python et mon maître de stage}
                \item Loïc Petit, \textcolor{red}{Java JBM}
                \item Clément Chastagnol, \textcolor{red}{data scientist Python et mon maître de stage}
                \item Clément Déon, \textcolor{red}{Front end développeur}
                \item Fabien Bréant, \textcolor{red}{Back end développeur}
                \item Jacques Belissent, \textcolor{red}{?}
                \item Vincent Ysmal, \textcolor{red}{Java / Back end développeur}
            \end{itemize}

        \begin{figure}[h!]
            \centering
            \begin{subfigure}[b]{0.2\textwidth}
                \includegraphics[width=\textwidth]{images/christian-serieux.png}
                \caption{Christian F.}
                \label{fig:christian}
            \end{subfigure}
            \begin{subfigure}[b]{0.2\textwidth}
                \includegraphics[width=\textwidth]{images/thomas2-Copier-Copier.jpg}
                \caption{Thomas D.}
            \end{subfigure}
            \begin{subfigure}[b]{0.2\textwidth}
                \includegraphics[width=\textwidth]{images/guillaume-serieux.png}
                \caption{Guillaume L.}
            \end{subfigure}
            \begin{subfigure}[b]{0.2\textwidth}
                \includegraphics[width=\textwidth]{images/samuel-serieux.png}
                \caption{Samuel C.}
            \end{subfigure}
            \begin{subfigure}[b]{0.2\textwidth}
                \includegraphics[width=\textwidth]{images/loic-serieux.png}
                \caption{Loïc P.}
            \end{subfigure}
            \begin{subfigure}[b]{0.2\textwidth}
                \includegraphics[width=\textwidth]{images/clement-c-serieux.png}
                \caption{Clément C.}
            \end{subfigure}
            \begin{subfigure}[b]{0.2\textwidth}
                \includegraphics[width=\textwidth]{images/clement-d-serieux.png}
                \caption{Clément D.}
            \end{subfigure}
            %\begin{subfigure}[b]{0.2\textwidth}
            %    \includegraphics[width=\textwidth]{images/jacques.jpg}
            %            \caption{Jacques B.}
            %\end{subfigure}
            \begin{subfigure}[b]{0.2\textwidth}
                \includegraphics[width=\textwidth]{images/vincent.jpg}
                \caption{Vincent Y.}
            \end{subfigure}
            \caption{L'équipe technique de Data Publica}
            \label{fig:teamd_data_publica}
        \end{figure}

\newpage

        \paragraph{L'équipe commerciale :}
            Elle est composé de 4 commerciaux (ordonnés par ancienneté visible en figure \ref{fig:teamc_data_publica}) :
            \begin{itemize}
                \item François Bancilhon, directeur général
                \item Benjamin Gans, Responsable Communication et Marketing
                \item Emmanuel Jouanne, Business Development Manager
                \item Philippe Spenato, Ingénieur d'affaire
                \item Justine Pourrat, Responsable Communication et marketing
            \end{itemize}

        \begin{figure}[h!]
            \centering
            \begin{subfigure}[b]{0.2\textwidth}
                \includegraphics[width=\textwidth]{images/francois-serieux.png}
                \caption{François B.}
                \label{fig:francois}
            \end{subfigure}
            \begin{subfigure}[b]{0.2\textwidth}
                \includegraphics[width=\textwidth]{images/philippe-1-serieux.png}
                \caption{Philippe S.}
            \end{subfigure}
            \begin{subfigure}[b]{0.2\textwidth}
                \includegraphics[width=\textwidth]{images/Justine-serieuse-crop.jpg}
                \caption{Justine P.}
            \end{subfigure}
            \caption{L'équipe commerciale de Data Publica}
            \label{fig:teamc_data_publica}
        \end{figure}

\section{C-Radar}\label{c_radar}
    \subsection{Présentation commerciale de C-Radar}
        Son produit est un moteur de recherche B2B (Business to Business). Celui-ci a pour objectif de permettre aux services ventes et marketing des entreprises B2B de vendre plus et mieux.\\
        Ce moteur de recherche, appelé C-Radar, est un produit de vente prédictive construit sur une base de référence des entreprises françaises. Il regroupe beaucoup d'informations sur les entreprises françaises, dont notamment leurs informations administratives, financières et toutes celles qui découlent de leur communication sur les réseaux sociaux et le web.\\

        C-Radar est un concentré de technologies du big data. En effet, il utilise diverses technologies comme le crawling, le scraping ou encore le machine learning. Ceci afin d'offrir à l'utilisateur diverses fonctionnalités : moteur de recherche d'entreprises, fiche d'activité d'entreprises avec contacts commerciaux, détection de nouveaux prospects, scoring de prospects existants, segmentation automatique d'entreprises, identification de marché, etc.

    \subsection{Présentation technique de C-Radar}
        \subsubsection{Les technologies utilisées par C-Radar}
            Pour répondre aux problématiques auxquelles Data Publica se confronte, la société a acquis 4 expertises majeures :
            \begin{itemize}
                \item Le web crawling / web scraping : la récupération des données ;
                \item Le data mining / text mining : l'analyse, l’extraction et l’enrichissement des données;
                \item Le machine learning : l'apprentissage automatique à partir de données structurées ;
                \item La dataviz : la visualisation des données.
            \end{itemize}

            \paragraph{Le crawling :}
                Le crawling est l’action réalisée par un programme informatique, appelé le web crawler, qui va de site en site afin d’en extraire automatiquement toute l’information qui est présente sur les différentes pages. Cette technique est utilisée notamment pour l’extraction de données non structurées: la structure du site n’est pas connue à l’avance, l’extraction des données se fait directement sur le contenu (c’est à dire le code HTML) de la page crawlée. Ce processus est \og brutal \fg.

            \paragraph{Le scraping :}
                Le scraping est l’action réalisée par un programme informatique pour extraire des unités d’information structurées d’un site web. Contrairement au crawling, il est question d’extraire des données précises, et pas la totalité des données disponibles sur le site. Le site “scrappé” et sa structure doivent donc être connus et analysés à l’avance afin d’adapter le scraper au site. Ce processus est \og intelligent \fg.

            \color{red}

            \paragraph{Le data mining / text mining :}
                Une fois des sites web crawlés et scrapés, ou que des flux (RSS ou réseaux sociaux) aient été captés, on peut commencer à analyser le contenu récupéré à la recherche d'informations sous forme de patterns particuliers, par exemple. On analyse afin de normaliser des problèmes d'encodages, de structures de date, de numéros de téléphone, etc.\\
                Globalement, cette phase consiste à regarder les données \og dans le fonds des yeux \fg \autocite{steph_canu} afin de voir leurs fonds mais aussi leurs formes.

            \paragraph{Le machine learning :}
                Quand les données sont correctement formatées et normalisées, on peut construire des applications capable d'apprendre automatiquement de ces données, de les classifier. Ainsi quand on aura une nouvelle donnée l'application sera capable de prédire sa classe d'après ses caractéristiques.\\
                L'idée derrière le machine learning est de pouvoir extraire automatiquement des informations d'une nouvelle donnée et ainsi prédire une classe de donnée.

            \paragraph{La dataviz :}
                C'est la dernière étape. Elle présente les données de manière visuelle et interprétable. Ainsi, on peut comprendre plus facilement et rapidement les informations extraites des données.


        \subsubsection{L'architecture technique de C-Radar}
            L'architecture de C-Radar peut être divisée en plusieurs parties (visible en figure \ref{fig:archi}) :
            \begin{itemize}
                \item Différentes bases de données pour différents stockages (une base de données Mongo, une base de données Cassandra et une base de données PostGreSQL) ;
                \item Un moteur de recherche sémantique (Elastic Search) ;
                \item Un gestionnaire de queue (RabbitMQ) ;
                \item Différents plugins Python s'interfaçant avec le gestionnaire de queue RabbitMQ ;
                \item Un gestionnaire de flux entre les différents composants précédents, le Workflow ;
                \item Une application Java s'interfaçant avec Elastic Search et les bases de données Mongo et PostGreSQL.
            \end{itemize}

            \paragraph{Le JBM ou Java Base Manager :}
                Le JBM est le projet Java qui rassemble le Workflow et l'application \href{app.c-radar.com}{app.c-radar.com}. Ce projet a été conçu et construit au dessus de Spring. Spring est un framework Java permettant de créer des applications web. Il prend en charge énormément de chose dont notamment le modèle MVC, la sécurité de l'application, l'interface avec les gestionnaire de queue, etc. (Pour plus d'information, voir \href{http://spring.io/}{http://spring.io/}).

            \paragraph{Le Workflow :}
                Le Workflow est le gestionnaire de flux permettant de lancer les différents processus de récupération et d'analyses des données. Il gère tout ce qui est \og computing \fg. Par exemple, c'est lui qui lance RabbitMQ qui lui même lance l'exécution des plugins Python gérant le processus de crawling de sites web, par exemple. Une fois les sites webs crawlés, le Workflow va les stocker dans Cassandra. Il gère tout ce qui est exécution des plugins Python (crawling et scraping du web, capture et catégorisation des signaux, etc), stockage des données produites à l'issue de ces exécutions et indexation dans Elastic Search.\\
                Pour résumer, il prépare les données que l'application va se charger de présenter à l'utilisateur.

            \paragraph{L'application \href{app.c-radar.com}{app.c-radar.com} :}
                L'application, \og présente \fg les données aux utilisateurs. Elle fait le reporting des données produites par le Workflow. Elle permet de rechercher des entreprises, de voir leur répartitions géographiques, de créer des listes d'entreprises, etc. C'est l'application visible et utilisée par l'utilisateur.

            \paragraph{Les bases de données :}
                Les bases de données stockent différentes données. La base Mongo stocke les données liées aux entreprises et les signaux, par exemple.

            \paragraph{Les plugins Python :}
                Enfin, les plugins Python sont des applications Python connectées à d'autres composants, afin d'exécuter une tâche sur des données. Ces données sont reçues des autres composants et le plugin leur retourne les résultats de sa tâche.

            \color{black}


            \begin{figure}[h!]
                \centering
                \includegraphics[width=\textwidth]{images/archi.jpg}
                \caption{L'architecture générale de C-Radar}
                \label{fig:archi}
            \end{figure}

\chapter{Présentation du sujet}
À la suite d'une candidature spontanée et de premiers contacts avec Thomas Dudouet et Clément Chastagnol, je me suis entretenu avec Christian Frisch afin de savoir quelle pourrait être ma contribution chez Data Publica. Ayant suivi une formation à l'INSA plutôt orienté big data / data mining, Christian Frisch m'a alors proposé de travailler sur l'une des fonctionnalité du produit C-Radar.

\section{La fonctionnalité de C-Radar} % (fold)
\label{sec:la_fonctionnalite_de_c_radar}
    L'objectif de mon stage serait le suivant : améliorer la fonctionnalité de C-Radar permettant d'être au courant de l'intégralité de la communication des entreprises françaises et belges simplement en s'abonnant à un newsletter.\\

    \begin{figure}[h!]
        \centering
        \includegraphics[width=0.75\textwidth]{images/capture_process.jpg}
        \caption{Le fonctionnement générale de cette fonctionnalité.}
        \label{fig:capture_process}
    \end{figure}

    De l'offre d'emploi, à la participation à des salons, en passant par les nominations de personnel ainsi que les présentations des derniers produits ou encore des éventuelles levées de fonds ou investissements, la communication des entreprises n'est plus équivoque mais belle et bien ordonnée grâce à cette fonctionnalité. On peut désormais savoir quelles sont les derniers postes à pourvoir chez Orange, par exemple, ou encore les dernière nomination qui ont eu lieux à la Banque Postale.\\

    Le fonctionnement générale de cette fonctionnalité est le suivant (visible en figure \ref{fig:capture_process}) :
    \begin{enumerate}
        \item C-Radar capte tout les signaux émis par les entreprises sur les réseaux sociaux (Facebook et Twitter), dans les médias ou via des flux RSS. Ces signaux sont ensuite stocké dans une base de données Mongo.
        \item Une fois ces signaux capturés et stocké, il faut les traiter afin d'identifier l'intérêt potentiel de leur contenu.
    \end{enumerate}
    C'est sur ce second point que j'interviens.
% section la_fonctionnalit_de_c_radar (end)

\section{Ma mission chez Data Publica} % (fold)
\label{sec:ma_mission_chez_data_publica}
    Ma mission est de construire une chaîne de traitement automatique, un plugin Python, récupérant la liste des signaux émis par les entreprises en entrée depuis un point d'API, leur appliquer les traitements nécessaires afin de fournir, en sortie, la liste des signaux intéressants ainsi que leur catégories respectives.\\

    Il s'agit donc de construire une application, un plugin Python, capable de classifier un signal dans une catégorie par la seule connaissance de son contenu (éventuellement un titre). L'application traitera donc des documents textuels.\\
    Le travail se divise en deux étapes :
    \begin{enumerate}
        \item Appliquer une série de prétraitements sur le contenu des signaux afin de sélectionner les features jugés porteurs d'information. En effet, comme les données manipulées sont textuelles, il faut filtrer certaines chaînes de caractères considérées comme du bruit.
        \item Construire un classifieur à partir des features sélectionnés parmi les données que l'on juge porteur d'information.
    \end{enumerate}

    \subsection{Inscription dans le domaine du \textit{Big Data}}
        Les tâches qui découlent de ma mission, et notamment tout les prétraitements, sont directement liées à la discipline de la fouille de textes (\textit{Text Mining}) (et de l'\textit{Information Retrieval}) pour trouver des informations dans le contenu des signaux. Elles sont également liées à la discipline du traitement automatique du langage naturel (\textit{Natural Language Processing}). Enfin, la construction du classifieur automatique implique des compétences en \textit{Machine Learning}.\\

    \subsection{Synthèse du travail à réaliser}
        Si l'on devait résumer le travail à effectuer (visible en figure \ref{fig:capture_process}), on pourrait le synthétiser comme ceci : Construire une application (plugin Python), qui prend en entrée un signal émis par une entreprise et répond aux questions suivantes :
        \begin{enumerate}
            \item Ce signal a-t-il de l'intérêt ?
            \item Si oui, quel est le sujet du signal ? Parle -t-il d'une offre d'emploi, d'une nomination, d'une levée de fond, etc ?
        \end{enumerate}
        La figure \ref{fig:process} montre ce que l'application (le plugin Python) doit être capable de faire.

        \begin{figure}[h!]
            \centering
            \includegraphics[width=\textwidth]{images/process.jpg}
            \caption{La tâche du plugin Python.}
            \label{fig:process}
        \end{figure}

\section{Présentation des signaux}
\label{sec:etat_bd}
    Les signaux sont des posts Facebook, des tweets ou bien des flux RSS publiés par des entreprises.

    \paragraph{Hypothèses de départ :}
        On considérera qu'un signal est intéressant si son contenu a pour sujet :
        \begin{itemize}
            \item Une offre d'emploi (ou un stage) à pourvoir au sein de l'entreprise qui l'a postée. Par la suite, on associera le tag \textit{JOB} à cette catégorie.
            \item Un évènement auquel l'entreprise participe (un salon par exemple). Par la suite, on associera le tag \textit{EVENT} à cette catégorie.
            \item Un produit que l'entreprise vient de présenter. Par la suite, on associera le tag \textit{PRODUCT} à cette catégorie.
            \item Une nomination d'un employé dans l'entreprise ou d'une entreprise vers une autre entreprise. Par la suite, on associera le tag \textit{PEOPLE} à cette catégorie.
            \item Une levée de fonds, un investissement, ou encore une déclaration de résultats ou de chiffre d'affaire.. Par la suite, on associera le tag \textit{MONEY} à cette catégorie.
        \end{itemize}

    \paragraph{Exemple de signal type pour chaque catégorie :}
        \begin{itemize}
            \item \textit{JOB} : \og Offre d'emploi: Ingénieur d'études et développement JAVA H/F (Orléans) http://t.co/LOvN8rLHIe \#jobs \fg
            \item \textit{EVENT} : \og  Fraispertuis sera présent dès demain jusqu'à dimanche inclus au salon \og Tourrissimo \fg de Strasbourg Parc des Expositions, Hall 21 Stand B40, venez rendre visite au capt'ain Fraisp ! \fg
            \item \textit{PRODUCT} : \og Découvrez quelques unes de nos réalisations de Pergola Biotempérée pour notre clientèle de Toulouse et sa région. Plus d'informations sur notre site www.pergola-biotemperee.com \fg
            \item \textit{PEOPLE} : \og Matthieu Frairot a été nommé Directeur Associé au sein de l'agence FullSIX France, l'agence marketin http://t.co/Z2ENeeWZmF \fg
            \item \textit{MONEY} : \og Keyrus : Publication des résultats annuels 2013. http://t.co/k4TPJ11fW4 \fg
        \end{itemize}

    \paragraph{État de l'ensemble des signaux :}
        Au 08/06/2015, seul 1426 signaux ont été validés manuellement avec potentiellement un tag (JOB, EVENT, PRODUCT, MONEY ou PEOPLE), dans le cas où le signal est intéressant.\\
        La proportion des classes de cet échantillon est mauvaise, c'est-à-dire que les données souffrent d'un fort déséquilibre entre classes : il y a beaucoup plus de signaux inintéressants qu'intéressants. Je n'avait donc pas assez d'exemples de signaux intéressants pour pouvoir considérer les éventuelles prédictions d'un classifieur construit sur la base de ces données comme correctes. Il est donc nécessaire d'en valider d'autres manuellement.

% section ma_mission_chez_data_publica (end)

\chapter{Travail effectué}
%Des travaux initiaux avait été réalisé en Python par Samuel Charron. Il avait créé un plugin Python capable de récupérer des signaux taggés depuis une API. N'étant pas formé au Python, j'ai préféré commencer mes travaux en utilisant le Java. Je savais en m'orientant vers le Java, qu'une fois que l'application obtiendrait de bonnes performance, j'aurais à implémenter son fonctionnement général en Python sous forme de plugin.

\section{Démarche de travail}
    Ce projet s'inscrit parfaitement dans le type de projet R\&D. De ce fait, l'avancement est très difficile à planifier dans le temps. Surtout lorsque l'on ne connaît pas les différentes notions sous-jacentes au projet et qu'il y a une bonne part d'auto-formation avant de pouvoir développer une application.\\

    \subsection{Mes acquis à l'INSA}
        Les connaissances générales que j'avais en \textit{Data Science}, avant le début du stage, concernaient le \textit{Data Mining} en contexte \textbf{numérique} et étaient les suivantes :
        \begin{itemize}
            \item Concepts en analyse et normalisation de données : Analyse en Composantes Principales (ACP), centrage et réduction de données numériques ;
            \item Concepts d'apprentissage non-supervisé : méthodes de regroupement des données (Clustering : Classification Hiérarchique Ascendante, Algorithme des K-Means, Modèles de mélanges et Algorithmes EM) ;
            \item Base de l'optimisation : méthodes du gradient et de Newton, introduction aux outils mathématiques pour l'optimisation sous contraintes convexe ;
            \item Concepts d'apprentissage supervisé : méthodes pour la discrimination de données (Décision Bayésienne, Régression logistique, SVM linéaire) et notions de validation croisée.\\
        \end{itemize}

\color{red}

        Mes connaissances en apprentissage non-supervisé ne m'étaient pas très utiles car ce projet est de type apprentissage supervisé. Cependant, les notions d'apprentissage supervisé telles que : la démarche à suivre pour construire un classifieur, les concepts liées à la validation des performances (validation croisée) et la notion de sur-apprentissage ; m'ont été très utiles.\\

        Malgré tout, mes connaissances n'étaient pas suffisamment étoffées pour pouvoir dire tel classifieur est plus performant qu'un autre dans tel contexte. En effet, mes connaissances sont axées manipulation et traitement de données en contexte \textbf{numérique}. De ce fait, la manipulation et le traitement de données textuelles m'étaient inconnus. De plus, je n'étais pas capable de choisir le meilleur classifieur que le contexte soit binaire ou multi-classes.\\

        Une formation en manipulation et traitement de contenu textuel m'a donc été indispensable avant de pouvoir commencer à travailler.

\color{black}

        \subsection{Formation en fouille de texte et en traitement automatique du language naturel}
            Dans ses travaux initiaux, Samuel Charron s'était intéressé au domaine du traitement automatique du langage naturel. Ainsi, il avait connaissance de l'existence de la bibliothèque de Stanford implémentée en Java (\textit{Stanford Natural Language Processing}). Il m'a donc conseillé de me former en fouille de texte et traitement automatique du language naturel au travers de celle-ci dans un premier temps. Une fois formé à ces domaines et à cette bibliothèque, je serais en mesure de construire un premier classifieur.\\

            Je me suis donc plongé dedans. J'ai commencé par lire les cours de l’université de Stanford sur le sujet, accessibles librement sur Internet. En parallèle, j'ai visionné sur coursera les vidéos que cette même université avait diffusé suite à un MOOC sur le traitement automatique du language naturel. Grâce à ces cours, j'ai pris conscience de toute l'importance du travail de prétraitement nécessaire à mettre en place, afin de bien normaliser et formater les données textuelles avant d'en faire quelque chose.\\

            Cette remarque prend tout son sens quand les données manipulées en plus d'être textuelles, ne sont pas normalisées (ou non structurées).\\

            Pour ma formation au traitement automatique du langage naturel, j'ai également lu les livres \autocite{nlp_p}, \autocite{nltk} et \autocite{ir_book} et les pages internets \autocite{ir_web}.

    \subsection{Déroulement du stage}
        Durant les deux premiers mois, j'ai exploré cette bibliothèque ainsi que les cours associés, et construit une première application Spring répondant aux contraintes évoquées en partie \ref{sec:ma_mission_chez_data_publica} (sauf le critère du langage : Python). Le travail en ressortant est décrit en partie \ref{sec:travaux_realises_en_java}.\\

        Ensuite, lors du dernier mois, j’ai implémenté le comportement général de cette application sous la forme d’un plugin Python, visible en partie \ref{}. J'ai ré-implémenté certain composants qui n'existaient pas en Python.

\section{Présentation des signaux}
    Les signaux sont des posts Facebook, des tweets ou bien des flux RSS publiés par des entreprises.

    \paragraph{Hypothèses de départ :}
        On considérera qu'un signal est intéressant si son contenu a pour sujet :
        \begin{itemize}
            \item Une offre d'emploi (ou un stage) à pourvoir au sein de l'entreprise qui l'a postée. Par la suite, on associera le tag \textit{JOB} à cette catégorie.
            \item Un évènement auquel l'entreprise participe (un salon par exemple). Par la suite, on associera le tag \textit{EVENT} à cette catégorie.
            \item Un produit que l'entreprise vient de présenter. Par la suite, on associera le tag \textit{PRODUCT} à cette catégorie.
            \item Une nomination d'un employé dans l'entreprise ou d'une entreprise vers une autre entreprise. Par la suite, on associera le tag \textit{PEOPLE} à cette catégorie.
            \item Une levée de fonds, un investissement, ou encore une déclaration de résultats ou de chiffre d'affaire.. Par la suite, on associera le tag \textit{MONEY} à cette catégorie.
        \end{itemize}

    \paragraph{Exemple de signal type pour chaque catégorie :}
        \begin{itemize}
            \item \textit{JOB} : \og Offre d'emploi: Ingénieur d'études et développement JAVA H/F (Orléans) http://t.co/LOvN8rLHIe \#jobs \fg
            \item \textit{EVENT} : \og  Fraispertuis sera présent dès demain jusqu'à dimanche inclus au salon \og Tourrissimo \fg de Strasbourg Parc des Expositions, Hall 21 Stand B40, venez rendre visite au capt'ain Fraisp ! \fg
            \item \textit{PRODUCT} : \og Découvrez quelques unes de nos réalisations de Pergola Biotempérée pour notre clientèle de Toulouse et sa région. Plus d'informations sur notre site www.pergola-biotemperee.com \fg
            \item \textit{PEOPLE} : \og Matthieu Frairot a été nommé Directeur Associé au sein de l'agence FullSIX France, l'agence marketin http://t.co/Z2ENeeWZmF \fg
            \item \textit{MONEY} : \og Keyrus : Publication des résultats annuels 2013. http://t.co/k4TPJ11fW4 \fg
        \end{itemize}

    \paragraph{État de la base de données contentant les signaux :}
        Au 08/06/2015, seul 1426 signaux ont été validés manuellement.
        La proportion des classes de cet échantillon est mauvaise, c'est-à-dire que les données souffrent d'un fort déséquilibre entre classes : il y a beaucoup plus de signaux inintéressants qu'intéressants. On avait donc pas assez d'exemples de signaux intéressants pour pouvoir considérer les éventuelles prédictions d'un classifieur construit sur la base de ces données comme correctes. Il est donc nécessaire d'en valider d'autres manuellement.

\section{Travaux réalisés en Java}
\label{sec:travaux_realises_en_java}
    \subsection{Présentation de mon environnement de travail}
        Pour pouvoir créer une application permettant de classifier les signaux, Loïc Petit m'a créé une application Spring de base permettant de se connecter en local à une base de données Mongo. Cette base de données Mongo sert à stocker un échantillon des signaux captés par C-Radar ainsi que les 1426 signaux validés.\\
        Je me suis rapidement formé à Spring pour pouvoir interfacer l'application Spring avec la base de données Mongo.\\

        Un signal capté est stocké au format JSON dans une base de données Mongo sous cette forme :
        \begin{verbatim}
        {
            "id" : "TWITTER:agencenetdesign:329129810423083009",
            "content" : "Salon eCom Genève : l'équipe ND est en place au
                         stand E1 :) #ecomSITB http://t.co/YtsEs6rcDR",
            "publicationDate" : ISODate("2013-04-30T07:07:16Z"),
            "sourceId" : "TWITTER:agencenetdesign",
            "source" : {
                "type" : "TWITTER",
                "resourceId" : "agencenetdesign"
            },
            "externalSignalId" : "329129810423083009",
            "validated" : false,
            "validatedTags" : [ ],
            "tags" : [
                "EVENT"
            ],
            "url" : "http://twitter.com/agencenetdesign/status/329129810423083009"
        }
        \end{verbatim}
        Le signal comporte :
        \begin{itemize}
            \item un identifieur unique \textit{id}
            \item un contenu \textit{content}
            \item une date de publication \textit{publicationDate}
            \item l'identifieur de la source \textit{sourceId}
            \item la source \textit{source} composé :
            \begin{itemize}
                \item du type de réseau dont provient le signal \textit{type}
                \item de l'identifieur du publieur dans ce réseau \textit{resourceId}
            \end{itemize}
            \item un identifeur externe \textit{externalSignalId}
            \item un booléen spécifiant si le signal a été manuellement validé ou non \textit{validated}
            \item la liste des tags s'il a été validé \textit{validatedTags}, la liste des tags potentiels (trouvés par le classifieur)
            \item l'url là où a été publié le signal
        \end{itemize}


        \subsubsection{Première application Spring}
            \paragraph{Contenu de la base Mongo et description des signaux :}
                Ma base de données locale contient environ 350.000 signaux au format JSON car Mongo est une base de données orientées documents.\\
            Parmi, ces signaux, 1426 signaux ont été validés manuellement avec potentiellement un tag (JOB, EVENT, PRODUCT, MONEY ou PEOPLE), dans le cas où le signal est intéressant.\\
            Une fois, les bases de Spring et Mongo acquises, j'ai pu construire une application réalisant ces actions :
            \begin{itemize}
                \item Récupérer les signaux stockés dans Mongo sous forme de liste ;
                \item Créer un ensemble de données à partir des signaux validés manuellement ;
                \item Diviser aléatoirement cet ensemble de données en deux ensembles (un pour entraîner le classifieur et un pour le tester) tout en gardant la proportion de chaque tag dans les deux ensembles ;
                \item la suite de la démarche...
            \end{itemize}
        Ainsi, j'ai construit une première application capable de récupérer les signaux sous forme de liste depuis une base Mongo en locale, de construire un ensemble de données dans lequel un signal ...blablabla

        \paragraph{Le QA ou Quality Assessment :}
            L'objectif du QA est de demander la contribution d'un maximum de personnes sur une tâche de validation manuelle pénible.\\
            Durant mon stage j'ai organisé plusieurs QA pour approfondir l'ensemble des signaux d’apprentissage et de test.

        \paragraph{Proportion de signaux intéressants :}
            La quantité de signaux n'ayant pas d'intérêt, ici, est énorme (plus de 70\%). De ce fait, une pré-sélection des signaux à valider est nécessaire.\\
            Ainsi, j'ai réutilisé le premier classifieur implémenté en Python, construit sur la base des 1426 signaux. Ce classifieur a permis de classifier des signaux non validés.\\
            Ce sont ces signaux classifiés par le classifieur Python qui ont été sélectionnés pour être validés manuellement. Notamment ceux appartenant aux catégories EVENT, JOB et PRODUCT.\\
            Pour ceux appartenant aux catégories MONEY et PEOPLE, ils ont été sélectionnés pour être validés à l'aide d'expressions rationnelles pour faire ressortir des termes tels que \og levée de fonds \fg, \og chiffre d'affaire \fg, \og nommer \fg, \og nomination \fg, etc.
            De cette manière 2574 nouveaux signaux ont été validés manuellement.\\

        Au 27.07.2015, il y avait donc 4000 signaux validés manuellement par un humain :
        \begin{itemize}
            \item 488 catégorisés EVENT soit 12,2\%
            \item 258 catégorisés JOB soit 6,4\%
            \item 118 catégorisés MONEY soit 3\%
            \item 83 catégorisés PRODUCT soit 2,1\%
            \item 49 catégorisés PEOPLE soit 1,3\%
            \item 3004 validés mais considérés comme inintéressant soit 75\%
        \end{itemize}

\section{Travail de prétraitement des données} % (fold)
\label{sec:travail_de_pretraitement}

    \subsection{Présentation de la bibliothèque \textit{Stanford Natural Language Processing}}
        La bibliothèque de Stanford propose un ensemble d'outils pour le traitement automatique du langage naturel de la langue anglaise, chinoise et espagnole.
        Ces différents outils sont les suivants :
        \begin{itemize}
            \item \textit{Stanford Parser} : permet de connaître la structure grammaticale d'une phrase (sujet, verbe, complément, etc);
            \item \textit{Stanford POS Tagger} (Part-Of-Speech) : permet de savoir la fonction grammaticale de chaque mot d'une phrase ;
            \item \textit{Stanford Named Entity Recognizer} : permet d'identifier les groupes de mots qui sont des noms de personne, d'entreprises, de gènes, etc ;
            \item \textit{Stanford Classifier} : permet de construire un classifieur automatique pour la catégorisation de texte ;
            \item \textit{Stanford Deterministic Coreference Resolution System} : permet de trouver les expressions qui font référence à une même entité ;
            \item \textit{Stanford CoreNLP} : permet de construire une suite de traitements automatiques (les traitements précédents) sur de l'anglais, du chinois, de l'espagnol
        \end{itemize}

    \subsection{Débuts avec la bibliothèque}
        Au départ, Samuel Charron m'a simplement demandé de réutiliser cette bibliothèque afin de créer un classifieur binaire (2 classes) permettant de classer les signaux relatifs aux offres d'emploi et de stage (soit le tag \textit{JOB}).\\

Des travaux initiaux avaient été réalisés en Python par Samuel Charron. Il avait réalisé un plugin récupérant les signaux, construisant un classifieur naïf bayésien multinomial avec et permettant de classifier de nouveaux signaux. cependant les performances n'étaient pas suffisantes. N'étant pas formé au Python, j'ai préféré commencer mes travaux en utilisant le Java avec l'accord de Samuel. Je savais, en m'orientant vers le Java, qu'une fois que l'application obtiendrait de bonnes performances, j'aurais à implémenter son fonctionnement général en Python sous forme de plugin pour pouvoir l'intégrer à l'architecture de C-Radar.

\section{Démarche de travail}
    Ce projet s'inscrit parfaitement dans le type de projet R\&D. De ce fait, l'avancement est très difficile à planifier dans le temps. Surtout lorsque l'on ne connaît pas les différentes notions sous-jacentes au projet et qu'il y a une bonne part d'auto-formation avant de pouvoir développer une application.

    \subsection{Mes acquis à l'INSA}
        Les connaissances générales que j'avais en \textit{Data Science}, avant le début du stage, concernaient le \textit{Data Mining} en contexte \textbf{numérique} et étaient les suivantes :
        \begin{itemize}
            \item Concepts en analyse et normalisation de données : Analyse en Composantes Principales (ACP), centrage et réduction de données numériques ;
            \item Concepts d'apprentissage non-supervisé : méthodes de regroupement des données (Clustering : Classification Hiérarchique Ascendante, algorithmes des K-Means et EM) ;
            \item Base de l'optimisation : méthodes du gradient et de Newton, introduction à l'optimisation sous contraintes convexe ;
            \item Concepts d'apprentissage supervisé : méthodes pour la discrimination de données (Décision Bayésienne, Régression logistique, SVM linéaire) et notions de validation croisée.\\
        \end{itemize}

        Ce projet ne permet pas de mettre mes connaissances en apprentissage non-supervisé en avant. Cependant, mes notions d'apprentissage supervisé telles que : la démarche à suivre pour construire un classifieur, les concepts liés à la validation des performances (validation croisée) et la notion de sur-apprentissage ; ont été fort utiles.\\

        Mes connaissances en \textit{Text Mining} n'étaient pas suffisamment étoffées pour pouvoir choisir le classifieur le plus adapté à un contexte donné (binaire ou multi-classe). En effet, ma formation (à l'INSA) est axée manipulation et traitement de données \textbf{numériques}. De ce fait, une formation en \textit{Text Mining} m'était donc indispensable avant de pouvoir commencer le développement d'une application.

    \subsection{Déroulement du stage}
        Ainsi, durant les deux premiers mois, j'ai exploré le domaine du \textit{Text Mining} et du \textit{Natural Language Processing} au travers de la bibliothèque de Stanford implémentée en Java (\textit{Stanford Natural Language Processing}). Conjointement, j'ai étudié les cours associés, et construit une première application Spring répondant aux contraintes évoquées en partie \ref{sec:ma_mission_chez_data_publica} (sauf le critère du langage). Le travail en ressortant est décrit en partie \ref{sec:travaux_realises_en_java}.\\

        Ensuite, lors du dernier mois, j’ai implémenté le comportement général de cette application sous la forme d’un plugin Python (partie \ref{sec:travaux_python}). Certains composants n'existaient pas en Python, je les ai donc ré-implémentés.
\section{Travaux réalisés en Java, le \textit{Text Mining} et le \textit{Natural Language Processing} avec la bibliothèque de Stanford}
\label{sec:travaux_realises_en_java}
    \subsection{Présentation de mon environnement de travail}
        Dans C-Radar tous les traitements de type \og computing \fg (calcul) sont réalisés en Python (cf partie \ref{subsub:archi_tech}). De ce fait, créer une application permettant de classifier les signaux, devrait être fait en Python sous la forme d'un plugin. Ayant fait le choix de commencer mes recherches en Java, il a fallu m'initialiser un environnement de travail un peu différent de l'environnement de travail Python.

        \paragraph{Spring et MongoDB :}
            Loïc Petit m'a créé une application Spring de base permettant de me connecter en local à une base de données Mongo. Ce projet Spring est géré avec Maven. L'application me fournit un environnement de travail dans lequel construire le classifieur à partir des signaux validés récupérés depuis une base de données Mongo. Cette base de données contient mon ensemble de signaux permettant de construire et tester mon classifieur. Les 1426 signaux validés y sont stockés ainsi que 350.000 autres signaux non validés (voir le dernier paragraphe de la partie \ref{sec:etat_bd}).\\

        Voici comment les signaux sont stockés dans Mongo :
\begin{verbatim}
{
    "id" : "TWITTER:agencenetdesign:329129810423083009",
    "content" : "Salon eCom Genève : l'équipe ND est en place au stand E1 \:) #ecomSITB http://t.co/YtsEs6rcDR",
    "publicationDate" : ISODate(2013-04-30T07:07:16Z),
    "sourceId" : "TWITTER:agencenetdesign",
    "source" : {
        "type" : "TWITTER",
        "resourceId" : "agencenetdesign"
    },
    "externalSignalId" : 329129810423083009,
    "validated" : false,
    "validatedTags" : [ ],
    "tags" : [
        "EVENT"
    ],
    "url" : "http://twitter.com/agencenetdesign/status/329129810423083009"
}
\end{verbatim}

        Le signal comporte :
        \begin{itemize}
            \item un identifieur unique \textit{id} ;
            \item un contenu \textit{content} ;
            \item une date de publication \textit{publicationDate} ;
            \item l'identifieur de la source \textit{sourceId} ;
            \item la source \textit{source} composée :
            \begin{itemize}
                \item du type de réseau dont provient le signal \textit{type} ;
                \item de l'identifieur du publieur dans ce réseau \textit{resourceId} ;
            \end{itemize}
            \item un identifeur externe \textit{externalSignalId} ;
            \item un booléen spécifiant si le signal a été manuellement validé ou non \textit{validated} ;
            \item la liste des tags s'il a été validé \textit{validatedTags}, la liste des tags potentiels (trouvés par le classifieur) \textit{tags} ;
            \item l'url là où a été publié le signal \textit{url}.
        \end{itemize}

        \paragraph{Formation Spring et Mongo :}
            Je me suis rapidement formé à Spring et Mongo pour pouvoir interfacer ces deux composants ensemble. Pour cela, j'ai suivi les tutoriels de Spring disponibles sur \href{https://spring.io/guides/gs/accessing-data-mongodb/}{https://spring.io/guides/gs/accessing-data-mongodb/}.\\
            Grâce aux tutoriels, j'ai appris à créer mes premiers points d'API permettant de faire des requêtes dans Mongo. Ces requêtes sont relativement basiques : récupérer tous les signaux sous forme de liste, récupérer uniquement les signaux validés (également sous forme de liste), savoir combien de signaux sont stockés dans ma base, etc. La documentation de Mongo disponible sur \href{https://docs.mongodb.org/manual/}{https://docs.mongodb.org/manual/} m'a également bien aidé. Les concepts de base de Mongo ne sont pas très compliqués à comprendre quand on a des notions de base de données.\\

        Dès que j'ai été capable de faire ce type de requête auprès de ma base de données, il fallait me concentrer sur le traitement des signaux, et donc construire un classifieur. C'est à ce moment que je me suis intéressé à la bibliothèque de Stanford.

        \subsection{La bibliothèque : Stanford Natural Language Processing}
            Samuel Charron m'a conseillé de me former en \textit{Text Mining} et en \textit{Natural Language Processing} au travers de cette bibliothèque implémentée en Java. Celle-ci propose un ensemble d'outils pour le traitement automatique de la langue anglaise, chinoise et espagnole.

            \subsubsection{Les cours de l'université de Stanford sur le \textit{Natural Language Processing}}
                Ces cours accessibles librement sur Internet, m'ont permis de découvrir de nombreux concepts (expliqués ci-après et en partie \ref{ssec:premiere_mise_en_appli}) :
                \begin{itemize}
                    \item Le modèle du \textit{bag of words} ;
                    \item Le modèle \textit{maximum entropy} ;
                    \item Les prétraitements comme la lemmatisation.
                \end{itemize}
                Ces cours proviennent du livre \textit{Introduction to Information Retrieval}\autocite{ir_web}.\\

                En parallèle, j'ai visionné sur coursera les vidéos que cette même université avait diffusé suite à un MOOC sur le \textit{Natural Language Processing}. Grâce à ces cours, j'ai pris conscience de l'importance de bien choisir son modèle de classifieur. De plus, j'ai également compris l'intérêt des prétraitements, permettant de normaliser et formater les données textuelles afin d'augmenter les performances et la capacité de généralisation du classifieur.\\

                Durant ma formation au \textit{Natural Language Processing}, j'ai également lu les livres \textit{Natural Language Processing with Python}\autocite{nlp_p} et \textit{Python 3 Text Processing with NLTK 3 Cookbook}\autocite{nltk}, ainsi que les pages internets \textit{Introduction to Information Retrieval}\autocite{ir_web}.

            \subsubsection{Stanford POS Tagger (Part-Of-Speech) :}
                Le \textit{POS Tagger} permet de savoir la fonction grammaticale de chaque mot d'une phrase. Celui-ci fonctionne aussi pour le français.

                \paragraph{Exemple :}
                    Entrée \og Gustave is the firstname of a very famous french architect.\fg\\
                    Sortie :
\begin{lstlisting}
K$\overbrace{Gustave}^{\highlight[cyan]{NNP}} \overbrace{is}^{\highlight[green]{VBZ}} \overbrace{the}^{\highlight[magenta]{DT}} \overbrace{firstname}^{\highlight[cyan]{NN}} \overbrace{of}^{\highlight[orange]{IN}} \overbrace{a}^{\highlight[magenta]{DT}} \overbrace{very}^{\highlight[yellow]{RB}} \overbrace{famous}^{\highlight[yellow]{JJ}} \overbrace{french}^{\highlight[yellow]{JJ}} \overbrace{architect}^{\highlight[cyan]{NN}}\overbrace{.}^{\highlight[gray]{.}}$W
\end{lstlisting}
                \textit{$\highlight[cyan]{NNP}$} signifie qu'il s'agit d'un nom propre singulier, \textit{$\highlight[green]{VBZ}$} signifie qu'il s'agit d'un verbe à la 3ème personne du singulier au présent, \textit{$\highlight[magenta]{DT}$} signifie qu'il s'agit d'un déterminant, \textit{$\highlight[cyan]{NN}$} signifie qu'il s'agit d'un nom commun singulier, \textit{$\highlight[orange]{IN}$} signifie qu'il s'agit d'une préposition ou d'une conjonction de subordination, \textit{$\highlight[yellow]{RB}$} signifie qu'il s'agit d'un adverbe et \textit{$\highlight[yellow]{JJ}$} signifie qu'il s'agit d'un adjectif.

            \subsubsection{Stanford Parser :}
                Le \textit{Parser} permet de connaître la structure grammaticale d'une phrase, à savoir quel(s) groupe(s) de mots forme(nt) le sujet, le verbe et le complément. Cet outils est une sur-couche du \textit{POS Tagger} puisqu'il réutilise son résultat.

                \paragraph{Exemple :}
                Entrée \og My internship was a rewarding experience.\fg\\
                Sortie :
\begin{lstlisting}
K(\textcolor{blue}{ROOT}W
  K(\textcolor{blue}{S}W
    K(\textcolor{cyan}{NP} (\textcolor{pink}{PRP}\$ My) (\textcolor{cyan}{NN} internship))W
    K(\textcolor{green}{VP} (\textcolor{green}{VBD} was)W
      K(\textcolor{cyan}{NP} (\textcolor{magenta}{DT} a) (\textcolor{yellow}{JJ} rewarding) (\textcolor{cyan}{NN} experience))W
    K(\textcolor{gray}{.} .)))W
\end{lstlisting}

            \subsubsection{Stanford Named Entity Recognizer :}
                Le \textit{Named Entity Recognizer} permet d'identifier les groupes de mots qui sont des noms de personnes, d'entreprises, de gènes, etc.

                \paragraph{Exemple :}
                Entrée \og François Bancilhon is the CEO of Data Publica, a Startup located in Paris.\fg\\
                Sortie :
\begin{lstlisting}
K$\highlight[magenta]{François}$W K$\highlight[magenta]{Bancilhon}$W is the CEO of K$\highlight[orange]{Data}$W K$\highlight[orange]{Publica}$W, a Startup located in K$\highlight[violet]{Paris}$W.
\end{lstlisting}
                Les mots surlignés en magenta comme $\highlight[magenta]{François}$ $\highlight[magenta]{Bancilhon}$ sont potentiellement des noms de personnes (\textit{$\highlight[magenta]{PERSON}$}). Ceux surlignés en orange comme $\highlight[orange]{Data}$ $\highlight[orange]{Publica}$ sont potentiellement des noms d'organisations (\textit{$\highlight[orange]{ORGANIZATION}$}). Enfin, les mots surlignés en violet comme $\highlight[violet]{Paris}$ sont potentiellement des noms de lieux (\textit{$\highlight[violet]{LOCATION}$}).

            \subsubsection{Stanford Classifier :}
                Le \textit{Classifier} permet de construire un classifieur automatique pour la catégorisation de texte. Un classifieur \textit{maximum entropy} et un naïf bayésien sont disponibles. Cet outils propose une classe (\textit{ColumnDataClassifier}) qui permet de construire n'importe quel type de classifieur, simplement en lui fournissant un fichier de configurations, et des données d'apprentissage et de test. Le soucis de cette classe est que les données doivent se conformer à un format bien précis.

            \subsubsection{Stanford CoreNLP :}
                Le \textit{CoreNLP} permet de construire une suite de traitements automatiques sur de l'anglais, du chinois, de l'espagnol. Un exemple de chaîne de traitement est visible en figure \ref{fig:coreNLP}. La particularité de cet outils, est qu'il permet de réutiliser les traitements précédents les uns à la suite des autres (sauf le \textit{Stanford Classifier}). De plus, il permet aussi d'appliquer des traitements supplémentaires à ceux énoncés jusque là. En effet, celui-ci permet de faire de la recherche morphologique telle que lemmatisation ou stemming. Le \textit{POS Tagger} et le \textit{Tokenizer} sont réutilisés pour ces traitements.

            \subsubsection{Lemmatisation et stemming :}
                La lemmatisation consiste à trouver les lemmes de chaque mot d'une phrase. Ce traitement est détaillé en partie \ref{ssubsec:travaux_globaux}. Ce traitement nécessite que la phrase soit préalablement découpée en token (les mots de la phrase) par un \textit{Tokenizer} et que chacun d'entre eux soit tagué par le \textit{POS Tagger}. Ainsi, en couplant la fonction grammaticale d'un mot avec un dictionnaire, il est possible de retrouver le lemme du mot.

                \paragraph{Exemple :}
                Entrée \og No better example than these could have been found.\fg\\
                Sortie :
\begin{lstlisting}
K$\overbrace{No}^{\highlight[yellow]{no}}\ \overbrace{better}^{\highlight[yellow]{good}}\ \overbrace{example}^{\highlight[cyan]{example}}\ \overbrace{than}^{\highlight[orange]{than}}\ \overbrace{these}^{\highlight[magenta]{this}}\ \overbrace{could}^{\highlight[green]{can}}\ \overbrace{have}^{\highlight[green]{have}}\ \overbrace{been}^{\highlight[green]{be}}\ \overbrace{found}^{\highlight[green]{find}} \overbrace{.}^{\highlight[gray]{.}}$W
\end{lstlisting}

                \paragraph{Le stemming :}
                Le stemming est un traitement assez similaire à la lemmatisation. Celui-ci consiste à trouver les stems et non les lemmes. Ce traitement est détaillé en partie \ref{ssubsec:travaux_globaux}.

                \paragraph{Exemple :}
                Entrée \og Les chevaliers chevauchent leur chevaux, le cavalier son cheval.\fg\\
                Sortie :
\begin{lstlisting}
K$\overbrace{Les}^{le}\ \overbrace{chevaliers}^{cheva}\ \overbrace{chevauchent}^{cheva}\ \overbrace{leur}^{leur}\ \overbrace{chevaux}^{cheva} \overbrace{,}^{,}\ \overbrace{le}^{le}\ \overbrace{cavalier}^{caval}\ \overbrace{son}^{son}\ \overbrace{cheval}^{cheva} \overbrace{.}^{.}$W
\end{lstlisting}

            \begin{figure}[h!]
                \centering
                \includegraphics[width=\textwidth]{images/coreNLP.jpg}
                \caption{Trois traitements sont assignés au \textit{pipeline} (entité de la classe \textit{StanfordCoreNLP}) : \textit{Tokenization}, \textit{POS-Tagging} et \textit{Lemmatization}. En sortie, une chaîne de caractère présente le résultat de chaque traitement en colonne.}
                \label{fig:coreNLP}
            \end{figure}

            \subsubsection{Premier bilan sur la bibliothèque}
                Dans un premier temps, mon objectif est seulement de réutiliser la partie \textit{Stanford Classifier} de la bibliothèque pour créer un classifieur binaire. Celui-ci doit être capable de classifier les signaux relatifs aux offres d'emploi et de stage (soit le tag \textit{JOB}). La classe \textit{ColumnDataClassifier} va mettre d'une grande aide, même si elle exige des données dans un format particulier et qu'elle ne gère pas certains prétraitements.\\

                Des traitements comme la lemmatisation ou le stemming seront utiles, lors de la normalisation des données. Pouvoir utiliser le \textit{StanfordCoreNLP} pour normaliser et sélectionner les features utiles serait l'idéal, avant de les fournir au classifieur pour construire un modèle.

        \subsection{Première mise en pratique de la bibliothèque de Stanford}
        \label{ssec:premiere_mise_en_appli}
            \subsubsection{Première application Spring}
                J'ai construis une application réalisant les actions suivantes (visible en figure \ref{fig:classif_building} et expliquée ci-après) :
            \begin{itemize}
                \item Récupérer les signaux stockés dans Mongo sous forme de liste ;
                \item Créer un ensemble de données à partir des signaux validés manuellement ;
                \item Diviser aléatoirement cet ensemble de données en deux ensembles (un pour entraîner le classifieur et un pour le tester) tout en gardant la proportion de chaque classe dans les deux ensembles ;
                \item Entraîner un classifieur binaire naïf bayésien ;
                \item Fixer les hyper-paramètres du classifieur par validation croisée pendant la phase d'apprentissage ;
                \item Évaluer la qualité du classifieur construit (l'erreur de généralisation) en calculant sa précision et son rappel sur un ensemble de données de test (qui n'ont pas \og été vues \fg jusqu'ici par le classifieur).
            \end{itemize}

            \begin{figure}[h!]
                \centering
                \includegraphics[width=\textwidth]{images/classifier_building.jpg}
                \caption{La construction du classifieur.}
                \label{fig:classif_building}
            \end{figure}

            \subsubsection{Le modèle de classifieur}
                Quel type de classifieur construire ? Un SVM ? Une régression logistique ? Un modèle naïf bayésien ?\\
                Dans la littérature de la classification de texte, comme la détection de spam dans les emails ou l'analyse des sentiments (savoir si un texte est critique ou élogieux), il est plutôt commun de construire des classifieurs naïfs bayésiens avec comme caractéristique la fréquence des mots. Ainsi, j'ai choisi de construire un tel classifieur pour catégoriser mes signaux. (C'est également ce qu'avait fait Samuel Charron en Python.)

            \subsubsection{Construction de l'ensemble de données}
                Ensuite, s'est posée la question de comment utiliser les données labellisées pour construire un ensemble de données pour l'apprentissage et la validation. Sur ce point, j'ai choisi de construire mon ensemble de données selon le modèle du \textit{bag of words}.\\

                Dans ce modèle, un texte (une phrase ou un document) est représenté comme un sac (\textit{bag}), un ensemble de ses mots (au sens mathématique), sans se préoccuper de la grammaire ou de l'ordre des mots, mais en gardant la multiplicité. Ce modèle est communément utilisé en classification de document, quand la fréquence, l’occurrence des mots est utilisée comme caractéristique. C'est le cas du modèle naïf bayésien.\\

                Ainsi, un signal est caractérisé par la liste des occurrences des mots de son titre et son contenu.

            \paragraph{Exemple :}
                Modélisation de deux signaux à l'aide du \textit{bag of words} :
\begin{lstlisting}
K$Offre\ d'emploi\ :\ Ingénieur\ d’études\ et\ développement\ Java.$W
K$Offre\ de\ stage\ :\ Classification\ de\ signaux\ entreprises\ Java\ ou\ Python.$W
\end{lstlisting}
                À partir de ces deux signaux, une liste de mot est construite à l'aide d'un tokenizer :
\begin{verbatim}
[ "Offre", "d'", "emploi", ":", "Ingénieur", "études", "et", "développement",
  "Java", ".", "de", "stage", "Classification", "signaux", "entreprises",
  "ou", "Python" ]
\end{verbatim}
                Celle-ci contient 17 mots distincts. En utilisant son index, on peut représenter chaque signal comme un vecteur de taille 17 :
\begin{lstlisting}
K$[\ 1,\ 2,\ 1,\ 1,\ 1,\ 1,\ 1,\ 1,\ 1,\ 1,\ 0,\ 0,\ 0,\ 0,\ 0,\ 0,\ 0\ ]$W
K$[\ 1,\ 0,\ 0,\ 1,\ 0,\ 0,\ 0,\ 0,\ 1,\ 1,\ 2,\ 1,\ 1,\ 1,\ 1,\ 1,\ 1\ ]$W
\end{lstlisting}
                Chaque ième composante du vecteur représente le nombre de fois que le ième mot de la liste est présent dans le signal. Par exemple, dans le premier vecteur (représente le premier signal), les deux premières composantes sont \og 1, 2 \fg. La première composante correspond au mot \og Offre \fg qui est le premier mot de la liste, et sa valeur est \og 1 \fg car \og Offre \fg est présent une fois dans le premier signal. De la même façon, la deuxième composante correspond au mot \og d' \fg, le deuxième mot de la liste et sa valeur est \og 2 \fg car il est présent deux fois dans le signal.\\

                Quelque soit le classifieur que j'ai construis par la suite, l'ensemble de données est toujours construit suivant ce modèle. Pour mon premier classifieur binaire, les 123 signaux validés \textit{JOB} de l'ensemble des 1426 signaux validés (présentés dans le dernier paragraphe de la partie \ref{sec:etat_bd}), forment la classe \textit{JOB} contre le reste. J'ai utilisé le \textit{PTB Tokenizer} de \textit{Stanford Classifier} (détails au paragraphe \textbf{La segmentation ou tokenisation} de la partie \ref{ssubsec:travaux_globaux}) pour découper les phrases en mots.\\

                Enfin, le seul pré-traitement des données mis en place est le fait de garder les éléments du \textit{bag of words} apparaissant au moins cinq fois. Ceci afin de supprimer les mots rares qui n'apportent pas d'informations, quand on fait de la classification textuelle (les mots mal orthographiés, les mots très spécifiques, les noms propres, etc).

            \subsubsection{Construction du classifieur}
                Pour construire mon classifieur binaire naïf bayésien, j'ai divisé mon ensemble de données en deux ensembles de données dans les proportions suivantes :
                \begin{itemize}
                    \item 75\% de l'ensemble de données pour l'ensemble d'apprentissage ;
                    \item 25\% de l'ensemble de données pour l'ensemble de test.\\
                \end{itemize}
                À noter que la méthode de la bibliothèque, permettant de diviser l'ensemble de données en deux ensembles, ne garantit pas que la proportion des classes soit conservée dans les deux ensembles résultant. De plus, l'ensemble n'est pas mélangé à chaque nouvelle construction du classifieur, il y donc un risque de sur-apprentissage. Christian Frisch me l'a fait remarqué, lors d'une réunion de suivi. J'ai donc repris le code source de la méthode pour ré-implémenter le bon comportement.

            \subsubsection{Optimisation des hyper-paramètres par validation croisée}
                Une méthode de la bibliothèque se charge d'optimiser l'hyper-paramètre du classifieur par validation croisée. Dans un premier temps, j'ai simplement réutilisé cette méthode, car les performances en étaient bonifiées.

            \subsubsection{Évaluation de la qualité du classifieur}
                Enfin pour évaluer la qualité de mon classifieur, je l'ai testé sur un ensemble de test, non utilisé jusqu'ici (les 25\% de l'ensemble initial). J'ai calculé la précision et le rappel obtenus par la classe \textit{JOB} et par la classe \textit{USELESS} représentant le reste (pas \textit{JOB}). Ceux-ci sont visibles en table \ref{tab:classif_perf}.
                \begin{table}[h]
                    \centering
                    \begin{tabular}{| c | c | c |}
                        \hline
                         & \textit{JOB} & \textit{USELESS} \\
                        \hline
                        Précision & 0,84 & 0,99 \\
                        Rappel & 0,91 & 0,98 \\
                        \hline
                    \end{tabular}
                    \caption{Performances du classifieurs.}
                    \label{tab:classif_perf}
                \end{table}

            \paragraph{Remarques :}
                Ici, on met l'accent sur le fait d'avoir une précision\footnote{Définition en annexe \ref{annexe:precision}} très élevée quitte à avoir un rappel\footnote{Définition en annexe \ref{annexe:rappel}} un peu diminué (car quand l'un augmente l'autre diminue et vice versa). On préfère se tromper très rarement dans notre classification (précision élevée) quitte à rater des signaux (rappel moyen). Autrement dit, on préfère qu'un utilisateur voit peu de signaux intéressants plutôt qu'il voit beaucoup de signaux dont l'intérêt reste à démontrer.

            \subsubsection{Premier bilan}
                Le classifieur construit est prometteur. Cependant, la généralisation à plusieurs classes, va amener à faire de nouveaux choix :
                \begin{itemize}
                    \item Faut-il adopter une stratégie de type \textit{One versus All} ou \textit{One versus One} ?
                    \item Est-ce que le modèle naïf bayésien va bien se généraliser ?
                    \item Faut-il un prétraitement global et/ou spécifique à chaque classe ?
                \end{itemize}

        \subsection{Amélioration de ma première application et prétraitement}
            La démarche de construction du classifieur présentée dans la partie précédente (\ref{ssec:premiere_mise_en_appli}) reste la même. Cependant, j'ai repris la classe principale \textit{ColumnDataClassifier} de la bibliothèque \textit{Stanford Classifier}, car elle intègre, notamment, un vrai processus de validation croisée en k plis. Je l'ai donc refactorée à plusieurs fins :
            \begin{itemize}
                \item Pour qu'elle récupère mes données directement depuis Mongo ;
                \item Pour qu'elle leur applique des prétraitements avant de construire l'ensemble de données ;
                \item Pour qu'elle réalise dix plis de validation croisée lors de l'apprentissage du classifieur.
            \end{itemize}
            Ce dernier point, me permet d'avoir une idée de la capacité de généralisation de mon classifieur.
            \subsubsection{Agrandissement de mon ensemble de signaux validés}
                Avant de pouvoir passer à un classifieur multi-classe, il fallait que mon ensemble de signaux validés grandisse. En effet, celui-ci souffre d'un fort déséquilibre entre les classes : il y a beaucoup plus de signaux inintéressants qu'intéressants. Ainsi, considérer les prédictions d'un classifieur construit avec ces données comme correctes était impossible. Il a donc été nécessaire d'en valider d'autres manuellement.\\

                Durant mon stage, j'ai donc organisé plusieurs QAs\footnote{Définition en annexe \ref{annexe:qa}.} (\textit{Quality Assessment}) pour approfondir l'ensemble des signaux d’apprentissage et de test.

                \paragraph{Proportions des signaux :}
                La proportion des signaux sans intérêt est énorme (plus de 80\% sur environ 300.000).\\
                Pour les QAs, j'ai donc fais une pré-sélection des signaux à valider. J'ai réutilisé le premier classifieur implémenté en Python de Samuel Charron, construit avec les 1426 signaux. Celui-ci a classifié des signaux non validés. Ce sont les signaux appartenant potentiellement aux catégories \textit{EVENT}, \textit{JOB} et \textit{PRODUCT} (\og selon le classifieur \fg) qui ont été sélectionnés pour être validés manuellement. Pour les classes \textit{MONEY} et \textit{PEOPLE}, ils ont été sélectionnés dans Mongo à l'aide d'expressions rationnelles sur des termes comme \og levée de fonds \fg, \og chiffre d'affaire \fg, \og nommer \fg, \og nomination \fg, etc.\\

                De cette manière, 2574 nouveaux signaux ont été validés. Ce type de méthode biaise les proportions des classes de signaux intéressantes par rapport à celle sans intérêt, mais autrement il mettait impossible d'obtenir plus d'exemple.\\

                Au 27.07.2015, il y avait donc 4000 signaux validés manuellement :
                \begin{itemize}
                    \item 488 catégorisés \textit{EVENT} soit 12,2\%
                    \item 258 catégorisés \textit{JOB} soit 6,4\%
                    \item 118 catégorisés \textit{MONEY} soit 3\%
                    \item 83 catégorisés \textit{PRODUCT} soit 2,1\%
                    \item 49 catégorisés \textit{PEOPLE} soit 1,3\%
                    \item 3004 validés mais considérés comme inintéressant soit 75\%
                \end{itemize}

            \subsubsection{Le passage au multi-classe}
                Pour le passage au multi-classe, la bibliothèque de Stanford impose une stratégie \textit{One vs All}. Concernant le modèle de classifieur, deux options s'offraient à moi :
                \begin{itemize}
                    \item Un modèle génératif\footnote{Définition en annexe \ref{annexe:generatif}} et notamment un classifieur naïf bayésien ;
                    \item Un modèle discriminant\footnote{Définition en annexe \ref{annexe:discriminatif}} et notamment un classifieur qui maximise l'entropie.
                \end{itemize}
                Jusque là, j'avais opté pour le modèle bayésien car c'est ce qui ressortait le plus souvent dans la littérature. J'ai donc repris mon application précédente et j'ai construit un classifieur naïf bayésien multinomial\footnote{qui suit plusieurs lois binomiales.} avec les 4000 signaux tagués de la même manière que précédemment. Les performances obtenues (visibles en table \ref{tab:classif_perf2}) avaient grandement baissées (la classe \textit{PRODUCT} n'y figure pas car elle n'était pas suffisament représentée).
                \begin{table}[h]
                    \centering
                    \begin{tabular}{| c | c | c | c | c | c |}
                        \hline
                         & \textit{JOB} & \textit{EVENT} & \textit{PEOPLE} & \textit{MONEY} \\
                        \hline
                        Précision & 0,63 & 0,51 & 0,46 & 0,44 \\
                        Rappel & 0,91 & 0,88 & 0,63 & 0,87 \\
                        \hline
                    \end{tabular}
                    \caption{Performances du classifieur naïf bayésien multinomial.}
                    \label{tab:classif_perf2}
                \end{table}

            \subsubsection{Le modèle discriminant maximisant l'entropie}
                Par curiosité, j'ai changé le modèle pour le modèle maximisant l'entropie. À ma grande surprise, ce modèle obtint de bien meilleures performances que le modèle bayésien dans les mêmes conditions (construction de l'ensemble de données identique, prétraitements identiques, etc). Les performances sont visibles en table \ref{tab:classif_perf3}.
                \begin{table}[h]
                    \centering
                    \begin{tabular}{| c | c | c | c | c | c |}
                        \hline
                         & \textit{JOB} & \textit{EVENT} & \textit{PEOPLE} & \textit{MONEY} \\
                        \hline
                        Précision & 0,96 & 0,82 & 0,73 & 0,81 \\
                        Rappel & 0,73 & 0,60 & 0,50 & 0,49 \\
                        \hline
                    \end{tabular}
                    \caption{Performances du classifieur maximisant l'entropie.}
                    \label{tab:classif_perf3}
                \end{table}

                \paragraph{Le \textit{MaxEnt} ou \textit{Maximum Entropy} :}
                    Les modèles \textit{maximum entropy} sont aussi connus sous le nom de \textit{softmax classifiers} et sont équivalents aux modèles de régression logistique multi-classe (avec des paramètres différents). Il s'agit de la généralisation de la régression logistique au cas multinomial. Dans ce cas, maximiser la vraisemblance revient à maximiser l'entropie. Il s'agit ni plus ni moins que du passage du cas binaire au cas N classes.\\
                    Ce type de modèle  (\textit{MaxEnt}) est avantageux dans le cas où les données sont \textit{sparse}\footnote{peu représentative, littéralement \og clairsemé \fg}. Ce qui est le cas, d'où des résultats meilleurs qu'avec le modèle bayésien.

            \subsubsection{Les travaux de prétraitement global}
            \label{ssubsec:travaux_globaux}
                Les traitements, expliqués ci-après, sont réalisés dans la construction de l'application, en amont de la construction de l'ensemble de donnée selon le modèle du \textit{bag of words}.

                \paragraph{La segmentation ou tokenisation :}
                    La tokenisation consiste à découper une phrase en token, dans l'idéal représentant des mots. Cette tâche est très spécifique à chaque langue. Les difficultés principales sont surtout liées aux contractions. Il est nécessaire de s'attarder sur cette phase de la normalisation pour minimiser la perte sémantique, car beaucoup de traitements s'appuient sur la tokenisation. Dans mon application, j'ai utilisé le \textit{PTB Tokenizer} disponible dans \textit{Stanford Classifier}.

                    \paragraph{Exemple :}
                    Entrée : \og Je n'ai pas d'argent. En as-tu ? \fg\\
                    Sorties possibles :
                    \begin{itemize}
                        \item ["Je", "n", "ai", "pas", "d", "argent", "En", "as", "tu"]
                        \item ["Je", "n", "'", "ai", "pas", "d", "'", "argent", ".", "En", "as", "-", "tu", "?"]
                        \item ["Je", "n'", "ai", "pas", "d'", "argent", ".", "En", "as", "-", "tu", "?"]
                    \end{itemize}
                    Le résultat du \textit{PTB Tokenizer} dans l'exemple précédant est le premier.

                \paragraph{La casse et la ponctuation :}
                    Pour normaliser les mots, une manière simple consiste à réduire la casse de tous les mots. Ainsi, on réduit notre ensemble de mot. Cela est très facile à effectuer mais cela a quand même quelques défauts :
                    \begin{itemize}
                        \item Les noms propres perdent leur différences par rapport aux noms communs ;
                        \item Les noms d'organisation (comme l'OTAN) perdent leur sens en minuscule.
                    \end{itemize}
                    De plus, supprimer la ponctuation permet aussi d'épurer les données car celle-ci n'apporte pas d'information. Encore une fois, c'est simple à réaliser mais certains mots perdent leur sens.
                    \begin{itemize}
                        \item Les mots composés comportant des tirets perdent leur sens (exemple : \og après-midi \fg);
                        \item Les mots composés comportant des apostrophes perdent leur sens (exemple : \og aujourd'hui \fg) ;
                        \item Les acronymes comportant des points de séparation perdent leur sens (exemple : \og U.S.A. \fg)
                    \end{itemize}

                \paragraph{Les stopwords :}
                    Les stopwords sont les mots d'une phrase inutiles à la compréhension de celle-ci. Ils ne portent pas d'information et sont présents dans n'importe quels documents textuels. Il est bien de les supprimer pour réduire le bruit. Une liste de stopwords contient majoritairement des pronoms, des prépositions et des déterminants comme : \og a \fg, \og au \fg, \og ce \fg, \og de \fg, \og le \fg, \og mon \fg, etc.

                \paragraph{La recherche morphologique :}
                    Enfin, les deux derniers moyens permettant de normaliser du texte sont la lemmatisation et le stemming. Ces deux traitements ont pour objectif de faire baisser le nombre de forme infléchie. Une forme infléchie est appelé un lexème en morphologie. Il faut savoir qu'un lexème est composé de différentes types de morphèmes : les stems et les affixes. Voici leur définition à l'aide d'un exemple :\\
                    Le lexème \og chanteurs \fg est composé de trois morphèmes : \og chant \fg, \og eur \fg et \og s \fg. Parmi ces trois morphèmes, l'un est un stem \og chant \fg et les deux autres des affixes \og eur \fg et \og s \fg.


                \paragraph{Le stemming :}
                    Le stemming réduit les lexèmes en leur stems : \og chanteurs \fg, \og chanteuse \fg transformés en \og chant \fg.\\
                    Un des désavantage du stemming est que les stems ne sont pas toujours des lemmes, c'est à dire la forme canonique d'un lexème (son entrée dans le dictionnaire), et donc pas un mot.\\
                    Exemple : le stem de \og chercheur \fg est \og cherch \fg (n'est pas un mot).

                \paragraph{La lemmatisation :}
                    La lemmatisation réduit les lexèmes en lemmes, la forme canonique d'un lexème. Le lemme d'un verbe correspond au verbe à l'infinitif, le lemme d'un nom commun est ce nom commun au masculin singulier, etc. Les lemmes correspondent aux entrées dans le dictionnaire de tous les lexèmes.\\
                    Exemple : le lemme de \og préférées\fg est \og préférer \fg.

                \paragraph{Remarques :}
                    Le stemming fait perdre plus d'information à nos features que la lemmatisation, j'ai donc choisi de garder les lemmes des mots.\\
                    La bibliothèque de Stanford ne propose pas de Lemmatizer pour la langue française. J'ai utilisé une \href{http://staffwww.dcs.shef.ac.uk/people/A.Aker/activityNLPProjects.html}{bibliothèque externe} de Ahmet Aker, que j'ai rajouté dans le projet Spring grâce à Maven en ajoutant une dépendance.

            \subsubsection{Bilan sur les prétraitements globaux}
                Dans l'idéal, j'aurais aimé pouvoir réaliser les prétraitements avec le \textit{Stanford CoreNLP} mais il ne gère pas bien le français et il est très gourmand en mémoire.\\
                En outre, les prétraitements permettent de normaliser le texte et de réduire le nombre de feature présent dans le sac de mot. Il est important de réduire la disparité des features pour pouvoir calculer leur fréquence par la suite.\\
                Un parallèle pourrait être fait entre la lemmatisation et la compression de données numériques par ACP (Analyse en Composantes Principales).

            \subsubsection{Les travaux de prétraitement spécifique}
                Il est possible et important de traiter nos données plus spécifiquement, afin d'augmenter les performances du classifieur par la suite. Pour cette tâche, il faut d'avantage s'attarder sur le fond des données de chaque classe, et se demander si des informations (tel que des motifs particuliers) ne seraient pas être détruites par les traitements précédents. Tous les traitements qui suivent sont réalisés en amont de la tokenisation.

                \paragraph{Les signaux issus de Twitter :}
                    Ceux-ci comportent souvent des références \og @pseudo \fg et des mentions comme \og \#job \fg. Les références n'apportent pas d'information dans la majorité des cas. Les supprimer permettrait d’éliminer du bruit. Quant aux mentions comme \og \#job \fg, celles-ci portent de l'information et dans un processus de normalisation, il serait bien de seulement garder le mot suivant le dièse (\#).

                \paragraph{Les emails et les URLs}
                    À l'image des pseudonymes Twitter, les emails et les URLs (présents dans certains signaux) ne portent aucune information et sont mal tokenisés du fait de leur formes. Ainsi, il serait bien de les supprimer.

                \paragraph{Les signaux de la classe \textit {JOB} :}
                    Souvent dans les offres d'emploi ou de stage, la mention \og H/F \fg signifiant \og homme ou femme \fg est présente. Cependant, lors de la tokenisation, ce type de motif est détruit à cause du caractère de ponctuation slash (/). Ainsi, il serait intéressant de pouvoir les détecter et de les remplacer par une chaîne de caractère sans ponctuation comme le terme \og hommeoufemme \fg.

                \paragraph{Les signaux de la classe \textit{MONEY}}
                    Une caractéristique des signaux de cette classe est que, souvent lorsque ces signaux parlent d'une levée de fonds, une somme est annoncée avec une devise. Exemple : \og L'INSA a levée 5 k€ pour construire un amphithéâtre \fg.\\
                    Il serait donc intéressant de conserver l'unité et le symbole de la devise suivant le montant qui est caractéristique de ce type de signal (k€, m€, etc).

                \paragraph{Mise en place de ces traitements et bilan :}
                    J'ai implémenté ces traitements assez facilement à l'aide d'expressions rationnelles. Ce travail d'inspection du contenu des signaux est important car c'est là que l'on détecte des informations caractéristiques.

            \subsubsection{Performances obtenues par mon application avec les prétraitements}
                Les performances, présentées en table \ref{tab:classif_perf4}, ont été obtenues en ajoutant tous les prétraitements présentés dans cette partie au processus de construction du classifieur.
                \begin{table}[t]
                    \centering
                    \begin{tabular}{| c | c | c | c | c | c |}
                        \hline
                         & \textit{JOB} & \textit{EVENT} & \textit{PEOPLE} & \textit{MONEY} \\
                        \hline
                        Précision & 0,96 & 0,82 & 0,94 & 0,81 \\
                        Rappel & 0,81 & 0,63 & 0,60 & 0,60 \\
                        \hline
                    \end{tabular}
                    \caption{Performances du classifieur maximisant l'entropie.}
                    \label{tab:classif_perf4}
                \end{table}


                \paragraph{Le problème des FP (faux positifs) :}
                    Les signaux qui peuvent avoir potentiellement plusieurs labels engendrent une quantité non négligeable de faux positifs, comme par exemple :
                    \begin{itemize}
                        \item Signal : "Venez découvrir nos nouveautés produits du 20 au 22 Mai 2014 au salon SEPEM, Hall 4 - Stand F13 http://t.co/vS7gT2x4yp \#marquagepermanent" Label : \textit{PRODUCT} ou \textit{EVENT}
                        \item Signal : "Nous embauchons! Étudiants de HEC Paris, nous sommes aujourd'hui au \#CarrefoursHEC. Venez découvrir nos offres d'emploi dans les domaines du \#digital \#data \#marketing" Label : \textit{JOB} ou \textit{EVENT}
                    \end{itemize}
                    Une solution potentielle serait d'autoriser le classifieur à attribuer plusieurs label. C'est ce qu'avait fait Samuel Charron en Python mais les résultats n'étaient pas bons.

                \paragraph{Le problème des FN (faux négatifs) :}
                    Les signaux ne sont pas toujours très succins et précis. De ce fait, parfois en plus d'un contenu intéressant le signal peut contenir du bruit, ce qui engendre une quantité non négligeable de faux négatifs. Ce qui baisse les rappels. Exemple :
                    \begin{itemize}
                        \item Signal : "Dans le cadre de son développement, EXCELIUM a fait l'acquisition du fonds de commerce de la société lyonnaise SES Vidéo, spécialisée dans l'installation de systèmes vidéo et d'alarme." Label : Aucun alors que c'est \textit{MONEY}
                        \item Signal : "A l'occasion de la fête de la gastronomie, nous vous invitons à visiter notre nouveau laboratoire à Téteghem samedi 27 septembre de 10 h à 18h. Visite, dégustation et chèques cadeaux offerts pour toute commande passée sur le site cette semaine !" Label : aucun alors que c'est \textit{EVENT}
                    \end{itemize}

                \paragraph{Performances du classifieur sur un ensemble de données non vus (obtenus grâce à un QA) :}
                    341 718 signaux ont été tagués automatiquement par le classifieur entraîné sur 4000 signaux validés.
                    (Certaines sources de signaux sont blacklistées et leurs signaux émis ne sont donc pas tagués par le classifieur : Les entreprises d'interim, les chasseurs de tête, les e-commerçants, etc). Parmi tous ces signaux, 388 ont été validés manuellement lors d'un QA. Ces signaux validés ont été sélectionnés aléatoirement parmi les 341 718 signaux tagués. Les performances du classifieur sur ces signaux sont visibles en \ref{tab:classif_perf4}.
                    \begin{table}[t]
                        \centering
                        \begin{tabular}{| c | c | c | c | c | c | c |}
                            \hline
                             & \textit{JOB} & \textit{EVENT} & \textit{PEOPLE} & \textit{PRODUCT} & \textit{MONEY} \\
                            \hline
                            Précision & 0,98 & 0,70 & 0,83 & 0,56 & 0,51 \\
                            Rappel & 0,92 & 0,80 & 0,97 & 0,75 & 0,95 \\
                            \hline
                        \end{tabular}
                        \caption{Performances du classifieur sur des données non validées.}
                        \label{tab:classif_perf4}
                    \end{table}

                Malgré tout, grâce à tout ces prétraitements, mon classifieur a obtenu de meilleures performances, jugées suffisantes par mon maître de stage Samuel Charron pour que je puisse passer à l'implémentation en Python. Les classes \textit{PRODUCT} et \textit{MONEY} manque d'exemple représentatif, d'où leur mauvais résultat. Un QA supplémentaire sur ces classes leur permettrait de se bonifier.
\section{Travaux réalisés en Python}
\label{sec:travaux_python}
    À la suite de ces deux mois de formations au \textit{Text Mining}, au \textit{Naturel Language Processing} et à l'\textit{Information Retrieval}, j'étais à présent capable d'identifier les tâches à réaliser en Python : Construire un module Python permettant de réaliser tout les prétraitements expliqués précédemment, reprendre le plugin Python de Samuel Charron afin d'appliquer ces prétraitement en amont de la construction de l'ensemble de données et enfin changer de modèle de classifieur.

    \subsection{Module de prétraitement spécifique aux signaux}
        Ce module de prétraitement (basé sur des expressions rationnelles) prend en charge la détection des patterns précédents à savoir : les emails, les URLs, les pseudonymes et les références Twitter, les mentions \og H/F \fg, les devises, la ponctuation et les stopwords ; et les traite spécifiquement :
        \begin{itemize}
            \item les emails, les URLs, les pseudonymes Twitter, la ponctuation et les stopwords sont supprimés ;
            \item les références Twitter perdent leur dièses (\#) ;
            \item les mentions \og H/F \fg  sont remplacés par la chaîne de caractère \og hommeoufemme \fg ;
            \item les devises (k€, m€) sont remplacés par les chaînes de caractère \og millier d'euros \fg, \og millions d'euros \fg.
        \end{itemize}

    \subsection{Module de lemmatisation}
        En Python 3, il n'existe pas de module permettant de faire de la lemmatisation de la langue française. Je me suis alors demandé comment le lemmatizer, que j'avais utilisé en Java, fonctionnait. Celui-ci couplait le \textit{POS Tagging} des mots à un autre outils, le HFST (\textit{Helsinki Finite State Transducer}), permettant de trouver les différents lemmes possibles d'un mot.\\

        \paragraph{Exemple d'utilisation du HFST:}
            Entrée : \og suis \fg\\
            Sortie :
\begin{lstlisting}
K$être+verb+singular+indicative+present+firstPerson$W
K$suivre+verb+singular+imperative+present+secondPerson$W
K$suivre+verb+singular+indicative+present+firstPerson$W
K$suivre+verb+singular+indicative+present+secondPerson$W
\end{lstlisting}

        \paragraph{Idée du lemmatizer Java:}
            \begin{enumerate}
                \item Pour chaque phrase, calculer le \textit{POS Tagging} de chaque mot via un POS Tagger ;
                \item Pour chaque mot de chaque phrase, trouver tous ses potentiels lemmes via le HFST ;
                \item Choisir le bon lemme grâce à l'information apporter par son \textit{POS Tagging}.
            \end{enumerate}

        \subsubsection{Helsinki Finite-State Transducer Technology (HFST)}
            Le \textit{Helsinki Finite-State Transducer software} est une implémentation de plusieurs outils d'analyse morphologique et d'autres outils basés sur les transducteurs (éventuellement pondérés) à état fini. Il s'agit d'un \href{http://www.ling.helsinki.fi/kieliteknologia/tutkimus/hfst/}{grand projet} portant sur les transducteurs littéralement en anglais \og transform reducer \fg.\\
            Le HFST propose un module Python permettant de trouver tout les lemmes d'un mot grâce notamment à un modèle de transducteur du français.

        \subsubsection{Stanford POST Tagger}
            Pour le POS Tagging, il existe un wrapper Python du \textit{Stanford POS Tagger}. Cet outils fournit également un modèle français pour taguer des phrases en français.\\

            \paragraph{Qualité du POS Tagger :}
                Pour vérifier la qualité de celui-ci, j'ai utilisé des corpus de texte français. Ceux-ci présentent un texte dans un format XML, sous forme de liste de paires mot/fonction grammaticale. J'ai du transformé le corpus en un texte composé de phrases afin de pouvoir le passer à mon \textit{POS Tagger} Python. J'ai également du transformé le corpus car les caractères de ponctuation étaient utilisé comme séparateur. Ceci créait un décalage entre la sortie de mon POS Tagging et cette liste de paires.\\
                Une fois les problèmes de formatage résolus pour un texte du corpus, j'ai testé le \textit{POS Tagger} dessus et celui-ci a obtenu un taux d'erreur jugé suffisant.

        \subsubsection{Lemmatiser Python :}
            J'ai donc mis au point Lemmatiser Python 3. Avec Clément Chastagnol, on a pu le déployé, mais un problème inattendu est survenu. En effet, à chaque appelle de la méthode de POS Tagging, derrière le wrapper Python lance la bibliothèque de Stanford dans une nouvelle JVM. Ceci ralenti énormément (facteur 100) le processus de traitement des signaux malgré une parallélisation des tâches (multithread).

    \subsection{Plugin de classification}
        Le plugin de classification permet de créer un modèle \textit{maximun entropy}, une régression logistique multinomiale, \textit{One vs All} à partir des 4000 signaux. Une fois le modèle créé, ce plugin est déployable et permet de taguer de nouveaux signaux. Celui-ci utilise les modules Python 3 scikit-learn et numpy.\\
        Numpy est un module pour la gestion de tableau (matrice).\\
        Scikit-learn est un module de machine learning construit au dessus de numpy pour faciliter la gestion des matrices.\\
        L'utilisation de scikit-learn est relativement facile car la documentation est très bien faite mêlant explications aux exemples d'utilisation.
\section{Bilan sur mes travaux}
    J'ai réussi à mettre au point un plugin Python fonctionnel, capable de classifier les signaux captés par C-Radar. Celui-ci obtient de meilleures performances que le précédent, ce qui était l'objectif, et celles-ci sont satisfaisantes. De plus, pour souligner l'aboutissement de ce stage, ce plugin devrait être déployé en production dans quelques semaines.\\

    Les principales difficultés rencontrées en première partie de stage sont liées à mon manque de connaissances en \textit{Text Mining} et \textit{Natural Language Processing}. J'ai cependant pu surmonter ces difficultés grâce à un bon accompagnement de la part de mes tuteurs et grâce à de l’expérience acquise, notamment, lors du projet de type R\&D pour Orange Vallée au semestre 8.\\

    J'ai rencontré d'autres difficultés. Celles-ci liées au le manque d'outils de \textit{Naturel Language Processing} disponible pour la langue française. En effet, pour l'anglais les outils sont très bien aboutis et nombreux, mais pour le français, il est difficile d'en trouver. Je pense, notamment, au Lemmatiser qui n'existe pas pour le français en Python 3. D'autres outils existent mais en Python 2. (De grandes différences existent entre Python 2 et Python 3. Parfois, des wrappers permettant de faire le pont entre les deux, mais leur utilisation ne sont pas de bonnes pratiques.)\\

    Les pistes d'amélioration possibles sont les suivantes :
    \begin{itemize}
        \item Introduire un outil permettant de détecter la langue du signal (comme \href{https://tika.apache.org/}{Tika}) afin d'adapter les traitements en aval ;
        \item Utiliser un détecteur d'entités nommées \textit{Named Entity Recognizer} afin de gérer autrement les noms d'entreprises, de personnes qui sont généralement caractéristiques des signaux \textit{MONEY} (exemple : \og Data Publica a levé 10 m€ en janvier auprès du fond d'investissement français constitué du groupe Lagardère...\fg). Pour le moment l'information apportée par ces noms propres n'est pas utilisée puisqu'ils ne sont pas retenus dans le vocabulaire du fait de leur rareté.
        \item Éventuellement, redéfinir la classe \textit{MONEY} en deux classes, comme suit, pour voir si le classifieur s'améliore :
        \begin{itemize}
            \item \textit{MONEY} : Déclaration de résultats financiers, de CA (chiffre d'affaire), de levée de fonds ;
            \item \textit{BUSINESS} : Déclaration de partenariat, de rachat ou d'aquisition d'entreprises, de \og contrat gagné\fg, etc.
        \end{itemize}
        \item Se tourner vers des outils statistiques comme le traducteur de Google pour exécuter la lemmatisation.\\
    \end{itemize}

\chapter{Conclusion}
Ce stage a été une expérience très enrichissante tant sur le plan humain et relationnel, que sur le plan technique. Je souhaitais absolument découvrir l'univers de la start-up et c'est chose fait grâce à Data Publica et ses équipes.\\

Sur le plan humain, j'ai eu la chance de rencontrer des personnes extrêmement sympathiques et compétentes, mais aussi ouvertes à la discussion et toujours disponible. L'équipe de Data Publica m'aura toujours bien encadré. Que ce soit Samuel Charron (et Christian Frisch) pendant la première partie stage ou Clément Chastagnol (et Guillaume Lebourgeois) en deuxième partie, j'aurais constamment eu quelqu'un à mon écoute. En effet, lorsque je rencontrais une difficulté, j'avais, à chaque moment, quelqu'un de disponible à \og déranger \fg pour mettre à l'épreuve son expérience ou ses connaissances.\\

Un excellent stage sur le plan relationnel aussi. En effet, tout le monde m'a accordé du temps à mon arrivée pour se présenter, et me dire leurs tâches et responsabilités chez Data Publica. De plus, partager des déjeuners et des pots conviviales a permis de renforcer et aussi de créer de premiers liens avec les collègues.\\

J'ai également beaucoup aimé cet esprit de partage et ces discussions ouvertes, lors des réunions des équipes pour différents points.\\

Sur le plan technique, j'ai énormément progresser dans le domaine du \textit{machine learning} et énormément appris dans le domaine du \textit{text mining} que je ne connaissais pas. Tous ces concepts de normalisation de textes, le langage Python et tous ces modules disponibles (scikit-learn, numpy, nltk, etc). J'ai découvert une facette de la \textit{data science} que j'ignorais et qui m'a beaucoup plu.//

Ce stage aura donc été très riche, techniquement parlant, par la découverte de domaines inconnus et passionnants, mais aussi humainement parlant, par la rencontre de personne formidable et poussant constamment à la réussite et au dépassement de soi.



\chapter{Résumé}
\input{parties/05-Resume.tex}

% Bibliographie
%\bibliographystyle{plain}
\printbibliography

% Annexes
\clearpage
\appendix
\chapter{Annexes}
\section{Les métriques de mesure de la qualité d'une classification}
    \subsection{Terminologies}
        Pour une classe donnée, voici la signification de ces terminologies :
        \begin{itemize}
            \item $\highlight[green]{TP}$ : True positive, bonne assignation à la classe ;
            \item $\highlight[green]{TN}$ : True negative, bon rejet de la classe ;
            \item $\highlight[red]{FP}$ : False positive, mauvaise assignation à la classe ;
            \item $\highlight[red]{FN}$ : False negative, mauvais rejet de la classe.
        \end{itemize}

    \subsection{La précision}
    \label{annexe:precision}
        Pour une classe donnée, la précision est le nombre de documents correctement assignés à cette classe rapporté au nombre de documents total assignés à cette classe par le classifieur.\\
        $Précision\ =\ \frac{TP}{TP+FP}$


    \subsection{Le rappel}
    \label{annexe:rappel}
        Pour une classe donnée, le rappel est défini par le nombre de documents correctement assignés à cette classe au regard du nombre total de documents appartenant réellement à cette classe.\\
        Le rappel mesure le fait que le classifieur ait trouvé tous les documents d'une classe.\\
        $Rappel\ =\ \frac{TP}{TP+FN}$

\section{Le QA ou Quality Assessment}
\label{annexe:qa}
    L'objectif du QA est de demander la contribution d'un maximum de personnes sur une tâche de validation manuelle pénible.\\

\section{Modèle de classifieur}
    Les points d'explication qui suivent, sont en partie tirés de Wikipédia\autocite{wiki_discri_gene}.
    \subsection{Le modèle génératif}
    \label{annexe:generatif}
        Le modèle génératif consiste à modéliser les probabilités conditionnelles soit $P(donnée | classe)$. Voici quelques exemples d'algorithmes de ce type :
        \begin{itemize}
            \item Classifieur naïf bayésien implique une distribution de probabilité conditionnelle de type binomiale (ou multinomiale) ;
            \item Analyse discriminante linéaire (LDA). Elle implique l'existence d'un modèle discriminant basé sur une distribution de probabilité de type Gaussienne.
        \end{itemize}
        Le modèle génératif maximise la vraisemblance de la probabilité jointe $P(classe, donnée)$.

    \subsection{Le modèle discriminant}
    \label{annexe:discriminatif}
        Le modèle discriminant cherche à maximiser la qualité de la classification. Une fonction de coût va réaliser l'adaptation du modèle de classification (en minimisant les erreurs). Voici quelques exemples d'algorithmes de ce type :
        \begin{itemize}
            \item Régression logistique, maximisation de la vraisemblance en considérant que les données suivent un modèle binomial ;
            \item Support Vectort Machine (SVM), maximisation de la marge entre l'hyperplan séparant les données.
        \end{itemize}
        Le modèle discriminatif maximise la vraisemblance de la probabilité conditionnelle $P(classe | donnée)$.

\section{Les transducteurs}
\label{annexe:transducteurs}
    Définition tirée de Wikipédia\autocite{wiki_trans}.\\

    En informatique théorique, en linguistique, et en particulier en théorie des automates, un transducteur fini (appelé aussi transducteur à états finis par une traduction maladroite de l'anglais finite state transducer) est un automate fini avec sorties. C'est une extension des automates finis. Ils opèrent en effet sur les mots sur un alphabet d'entrée et, au lieu de simplement accepter ou refuser le mot, ils le transforment, de manière parfois non déterministe, en un ou plusieurs mots sur un alphabet de sortie. Ceci permet des transformations de langages, et aussi des utilisations variées telles que notamment l'analyse syntaxique des langages de programmation, et l'analyse morphologique ou l'analyse phonologique en linguistique.\\

    D'autres explications détaillées sont disponibles \href{http://sixty-north.com/blog/deriving-transducers-from-first-pr}{ici}.


% pageQuatriemeCouverture
% #1 : Direction
% #2 : telephone
% #3 : e-mail
% #4 : Resume Haut
% #5 : Resume Bas
\pageQuatriemeCouverture{Département ASI}{02 32 95 97 79}{asi@insa-rouen.fr}{Ce stage obligatoire s'effectue en fin de quatrième année. Au cours de ce stage l'étudiant devra mettre en pratique les connaissances acquises au cours de sa formation et devra approfondir son savoir-faire au sein de l'entreprise. Il faudra qu'à  la fin de son stage l'étudiant réalise un rapport écrit. La validation du stage dépend de la qualité du travail réalisé, du rapport et de la fiche d'évaluation du tuteur industriel.\\
Ce stage a eu lieu chez Data Publica, un des précurseurs de l'open data en France. Data Publica est une jeune start-up (fondée en juillet 2011) spécialisée dans les données entreprises, l'open data, le big data et la dataviz. Data Publica emploie quatorze personnes très dynamiques et compétentes. Data Publica développe C-Radar, un produit de vente prédictive construit sur une base de référence des entreprises françaises regroupant informations administratives, financières, web, réseaux sociaux et media. C-Radar est un concentré de technologies du big data (crawling, scraping ou encore machine learning).\\
Ma mission, chez Data Publica, est de construire une chaîne de traitement automatique, un plugin Python, récupérant une liste de documents textuels en entrée et fournissant, en sortie, une liste de ces documents labellisés selon leur catégorie d'intérêt (offre d'emploi, participation à des événements, nomination de personnel, levée de fond, etc).\\
Les tâches qui en découlent sont directement liées au \textit{Text Mining} et au \textit{Natural Language Processing} (aussi à l'\textit{Information Retrieval}) dans la découverte d'informations dans le contenu des documents. Des compétences en \textit{Machine Learning} sont également requises.\\
Ce stage s'est conclue de belle manière, puisque le plugin Python est fonctionnel et capable de classifier de tels documents.}{This internship is compulsory and carried out at the end of the fourth year. During this internship, the student has to put into practice the knowledge acquired during his training and should improve his skills whithin the company. At the end of the intership, the student must write a report. The internship's validation depends on the quality of the work, the report and the evaluation sheet of the industrial tutor.\\
This internship took place at Data Publica, an open data pioneer in France. Data Publica is a young start-up (founded in July 2011) specialized in business data, open data, big data and DataViz. Data Publica employs fourteen dynamic and competent people. Data Publica is developing C-Radar, a predictive selling product built on French companies informations, like administrative, financial, web, social networks and media. C-Radar is made of big data technologies (crawling, scraping or machine learning).\\
My mission at Data Publica, was to build an automatic processing chain, a Python plugin, to which you give a list of textual records as an input. In return, you get a list of tagged documents according to their subject interest (job offer, event participation, staff nomination, fundraising, etc).\\
The directed tasks linked to the detection of information in the records are Text Mining and Natural Language Processing (also Information Retrieval). Skills in Machine Learning are also required.\\
My internship missions were all completed, the Python plugin is running and able to classify records.}



\end{document}
